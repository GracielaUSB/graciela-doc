\section*{ }

\begin{frame}{Tipos de Dato Abstractos\footfullcite{dalewalker}\footfullcite{ravelo}}

\begin{itemize}
  \item Recomendación más importante de Araujo y Jiménez.
  \item Centro de la presente extensión al lenguaje Graciela.
  \item Se busca la misma expresividad del lenguaje GCL.
  \item Restricciones y comportamientos verificados a tiempo de ejecución.
\end{itemize}
\end{frame}

\section{Tipos de Dato Abstractos}

\subsection*{Abstracciones}

\begin{frame}{Comportamientos abstractos}
\begin{itemize}
  \item Se definen con la palabra reservada \ingra{abstract}.
  \item Variables internas para establecer propiedades.
  \item Invariante de representación (\ingra{repinv}).
  \item Procedimientos y funciones sin cuerpos, definen una interfaz.
\end{itemize}
\end{frame}

\subsection*{Implementaciones}
\begin{frame}{Implementaciones}
Para aprovechar la definición de un tipo de dato abstracto, es necesario proveer una \textit{implementación} de éste.
\begin{itemize}
  \item Se definen con la palabra reservada \ingra{type}.
  \item Variables internas de \textbf{tipos básicos}, apuntadores, o arreglos.
  \item Invariante de representación (\ingra{repinv}).
\end{itemize}
\end{frame}

\begin{frame}{Implementaciones, cont.}
\begin{itemize}
    \item Se decidió separar el invariante de acoplamiento de la literatura en dos partes
    \begin{itemize}
      \item \textit{relación} de acoplamiento (\ingra{where})
      \item \textit{invariante} de acoplamiento (\ingra{coupinv}).
    \end{itemize}
  \item Por último, una \textit{implementación} también cuenta con procedimientos y funciones. 
  \item Si el TDA especifica un procedimiento o función, \textit{cualquier} \ingra{type} que busque implementarlo debe contar con su propia definición de éste.
\end{itemize}
\end{frame}

\begin{frame}{Tripleta de Hoare}
Si se tiene el procedimiento $\textbf{proc}\ p (Y)$, e $Inv(x)$ es el predicado que condensa todos los
invariantes del valor $x$ si $x$ es de un TDA, entonces se debe cumplir que

\begin{equation*} \label{eqn:tdatriple}
\begin{gathered}
% \frac{
  \{ P[X/Y] \land (\forall x\ |\ x \in X \land isTDA?(x)\ |\ Inv(x) )\}\\
  p\ (X)\\
  \{ Q[X/Y] \land (\forall x\ |\ x \in X \land isTDA?(x)\ |\ Inv(x) )\}
% }{
% }
\end{gathered}
\end{equation*}
\end{frame}

\begin{frame}{Relación \textit{implementa a}}
Por lo tanto, cuando en Graciela se hace una llamada, se genera código para las siguientes aserciones, en orden

\begin{enumerate}
  \item Precondición de la implementación.

  \item Relación e Invariante de acoplamiento.

  \item Invariante de representación de la \textit{implementación}.

  \item Precondición del TDA, si existe.

  \item Invariante de representación del TDA.

  \item \textit{Cuerpo del procedimiento}.


\end{enumerate}
\end{frame}

\begin{frame}{Relación \textit{implementa a}, cont.}
\begin{enumerate}\setcounter{enumi}{6}
  \item Relación e Invariante de acoplamiento.

  \item Invariante de representación del TDA.

  \item Postcondición del TDA, si existe.

  \item Invariante de representación de la \textit{implementación}.

  \item Postcondición de la implementación.
    \begin{itemize}
      \item Si se cumplieron ambas postcondiciones, la ejecución sigue normalmente
      \item Si sólo se cumple la postcondición del la implementación, se emite el
      error <<No se cumple la postcondición>>
      \item Si sólo se cumple la postcondición del TDA, se emite el error <<Este
      procedimiento no implementa el TDA>>
      \item Si ninguna de las dos postcondiciones se cumple, se emiten ambos errores.
    \end{itemize}
\end{enumerate}
\end{frame}

\subsection*{Implementación}

\begin{frame}{Detrás de la escena}
Los invariantes de representación (\ingra{repinv}), y los invariantes  (\ingra{coupinv}) y relaciones de acoplamiento (\ingra{where}) funcionan como rutinas internas que son generadas para cada TDA y cada implementación, y son llamadas en los momentos apropiados. 

Adicionalmente, para cada implementación, se generan rutinas internas para crear, copiar y eliminar objetos de su tipo. 
\end{frame}

\begin{frame}{Modos de parámetros}
Por supuesto, debe ser posible pasar objetos de un TDA en los distintos modos de parámetros del lenguaje Graciela, a saber, In, In-Out, Out y Ref.

\begin{description}
  \item [In]     Se crea en el heap una estructura del tipo apropiado, se copia el parámetro real en ella, pasando un apuntador a esta estructura.
  \item [Out]    Se crea en el heap una estructura del tipo apropiado, vacía, y se pasa un apuntador a esta estructura. Al finalizar el procedimiento, se copia esta estructura sobre el parámetro real.
  \item [In-Out] Combinación de los dos modos anteriores. 
  \item [Ref]    Simplemente se pasa una referencia al parámetro real.
  \item [Const]  Sólo disponible para variables de tipos básicos. 
\end{description}

\end{frame}

\defverbatim[colored]\typevar{
\begin{lstlisting}[language=graciela, style=code, escapechar=\~]
abstract ALista (A)
    ...

type Lista(B) implements ALista(B)
    ...
    proc p (in l : Lista(B))
      ...
\end{lstlisting}
}

\begin{frame}{Variables de tipo}
Una situación bastante frecuente en los TDA es que sus comportamientos son genéricos sobre los tipos que contienen. Por lo tanto, se decidió permitir
especificar los TDAs y sus implementaciones con variables de tipo. 

\typevar

En Graciela, estas variables de tipo siempre serán instanciadas en tipos básicos, y los procedimientos definidos dentro de \ingra{ALista(A)} y \ingra{Lista(B)} no pueden hacer suposiciones sobre \ingra{A} o \ingra{B} más allá de que son comparables, mostrables y legibles por teclado. 
\end{frame}

\defverbatim[colored]\typevarr{
\begin{lstlisting}[language=graciela, style=code, escapechar=\~]
var miListaInt  : Lista (int);
var miListaChar : Lista (char);

p(miListaInt);
p(miListaChar);
\end{lstlisting}
}

\begin{frame}{Variables de tipo, cont.}
Si se crea una variable de un TDA dentro del programa principal, o dentro de un procedimiento en el alcance global, se deben especificar tipos básicos para
cada una de las variables de tipo dadas en la declaración \ingra{type}.

\typevarr

A tiempo de compilación, se generan dos tipos de estructura, y para cada una, 
una copia de cada procedimiento, función, y rutina interna, cambiando únicamente
el tipo que sustituye a la variable de tipo \ingra{B}. Esto podría parecer 
exagerado, pero de hecho es la misma técnica de plantillas que utiliza el 
compilador de C++.
\end{frame}

\defverbatim[colored]\typevarrr{
\begin{lstlisting}[language=graciela, style=code, escapechar=\~]
type Lista(B) implements ALista(B)
    ...
    proc q (in l : Lista(B), in b : B)
      ...

var miListaInt  : Lista (int);
var miListaChar : Lista (char);

q (miListaInt,   3 )
q (miListaChar, '3')
\end{lstlisting}
}

\begin{frame}{Variables de tipo, cont.}
Por último, la verificación de tipos para llamadas a procedimientos de TDAs 
es capaz de reconocer el tipo apropiado para el segundo parámetro de un procedimiento como

\typevarrr
\end{frame}
