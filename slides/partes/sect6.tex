\section{ }

\section{Resultados}
\begin{frame}{Resultados}
\begin{itemize}
  % \item Cambios en la gramática. Bloque principal, postcondición y aserciones.
  \item Verificaciones para arreglos. Arreglos multidimensionales.
  \item Cotas en subrutinas
  \item Cuantificaciones más flexibles.
  \item Manejo manual de apuntadores.
  \item Teoría de conjuntos.
  \item Tipos de Dato Abstractos e Implementaciones.
  \item Apoyo al programador.
\end{itemize}
\end{frame}

\section{Conclusiones}
\begin{frame}{Conclusiones}
\begin{itemize}
  \item Para el \textit{Front-End} se utilizó la biblioteca de análisis sintáctico \textit{Megaparsec}\footfullcite{megaparsec}.
  \item Para el \textit{Back-End}
  se aprovechó la familia de software LLVM, a través de las bibliotecas \textit{llvm-general}\footfullcite{llvm-general} y \textit{llvm-general-pure}\footfullcite{llvm-general-pure}.
  \item El producto final de este proyecto es un lenguaje, junto con su compilador, diseñado para la enseñanza de los cursos de Algoritmos y Estructuras I y II de la USB.
\end{itemize}
\end{frame}

\section{Recomendaciones}
\begin{frame}{Recomendaciones}
\begin{itemize}
  % \item Compilación de varios archivos.
  % \item Tipos enumerados.
  \item Tipos Algebráicos Libres.
  % \item Escribir la biblioteca externa en Graciela, en lugar de en C++.
  % \item Colecciones de colecciones.
  \item Variables de tipo instanciadas en tipos no-básicos.
  % \item Subrutinas correcursivas (tupla de cotas).
  \item Lógica de separación.
\end{itemize}
\end{frame}