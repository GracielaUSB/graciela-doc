\section{ }

\begin{frame}{Teoría de Conjuntos}
\framesubtitle{Requisito indirecto}
Se hizo necesario proveerlos para facilitar la escritura de aserciones para los Tipos de Dato Abstractos.

\begin{itemize}
  \item{\makebox[2.5cm][l]{Conjuntos      } \ingra|\{ \}| }
  \item{\makebox[2.5cm][l]{Multiconjuntos } \ingra|\{: :\} ó | \Lbag \Rbag}
  \item{\makebox[2.5cm][l]{Secuencias     } \ingra|<< >> ó | \Lseq \Rseq}
  \item{\makebox[2.5cm][l]{Relaciones     } \ingra{rel()} }
  \item{\makebox[2.5cm][l]{Funciones      } \ingra{func()} }
\end{itemize}
\end{frame}

\section{Teoría de Conjuntos}

\subsection*{Algo de código}

\defverbatim[colored]\setops{
\setstretch{1.15}
\begin{lstlisting}[language=graciela, style=code, escapechar=\~]
// Operadores

~\Elem~, elem
~\Notelem~, notelem
!=
==
~\Subsett~, subset
~\Ssubset~, ~\SsubsetAlt~, ssubset
~\Superset~, superset
~\Ssuperset~, ~\SsupersetAlt~, ssuperset
\
~\Intersect~, intersect
~\Union~, union
~\Msum~, msum
#
~\Append~, ++
[]
()
\end{lstlisting}
}

\defverbatim[colored]\setfuncs{
\setstretch{1.15}
\begin{lstlisting}[language=graciela, style=code, escapechar=\~]
// Funciones

toSet()
toMultiset()
rel()
func()
multiplicity()
domain()
codomain()
\end{lstlisting}
}

\begin{frame}{Operadores y funciones}
\begin{columns}
\column[t]{5cm}
\begin{minipage}{0.48\textwidth}
\setops
\end{minipage}
\column[t]{5cm}
\begin{minipage}{0.48\textwidth}
\setfuncs
\end{minipage}
\end{columns}
\end{frame}


\defverbatim[colored]\settheory{
\setstretch{1.4}
\begin{lstlisting}[language=graciela, style=code, escapechar=\~]
var m := ~\Lbag~3, 3, 4, 4, 5~\Rbag~ ~\Msum~ ~\Lbag~5, 5, 7~\Rbag~ : multiset of int;
{ m == ~\Lbag~ 3, 3, 4, 4, 5, 5, 5, 7 ~\Rbag~ }
var i := multiplicity (4, m) : int;
{ i == 2 }
var s := toSet (m) ~\Union~ {~~3~~} : set of int;
{ s == {~~3, 4, 5, 7~~} }
var q := ~\Lseq~'a', 'b', 'c'~\Rseq~ : sequence of char;
var k := q[1] : char;
{ k == 'b' }
var d : function int -> char;
    d := func ({ (i, k), (~~2~~*i, q[0])});
var e : relation int <-> char;
    e := rel  ({ (i, d(i)), (i, q[0]), (i, 'z') });
var a := e (i) ~\Intersect~ {'z'} : set of char;
{ a == { 'z' } }
\end{lstlisting}
}

\begin{frame}{Ejemplos}
\settheory{}
\end{frame}

\subsection*{Implementación}

\begin{frame}{Biblioteca Externa}
\begin{itemize}
  \item Biblioteca externa escrita en C++\footfullcite{cpp-lang}.
  \item Aprovecha la biblioteca estándar \texttt{stdlibc++}.
  \item El manejo de memoria es responsabilidad del compilador.
  \item Se libera el espacio asignado cada vez que salen de alcance.
\end{itemize}
\end{frame}
