\section{Lo profundo}
\subsection{Objetivos}
\begin{frame}{Objetivos}
\begin{itemize}
  \item Tipos de datos estructurados abstractos
  \item Tipos de datos estructurados
  \item Apuntadores
\end{itemize}
\end{frame}


\subsection{Tipos de datos estructurados abstractos}
\begin{frame}[fragile]{La gramática}
\scriptsize
\begin{grammar}

<ADT> ::= `abstract' <Id> <TypeVars> `begin' <AbstBody> `end'

<TypeVars> ::= `(' <Id> <MoreIds> `)' | $\lambda$

<MoreIds> ::= `,' <Id> <MoreIds> | $\lambda$

<AbstBody> ::= <AbstVarDecs> <Invariant> <ProcDecs>

<AbstVarDecs> ::= <AbstVarDec> `;' <AbstVarDecs> | $\lambda$

<AbstVarDec> ::= (`var'|`const') <Id> <MoreIds> `:' <AbstType>

<AbstType> ::= `set' `of' <Basic'>
\alt `multiset' `of' <Basic'>
\alt `seq' `of' <Basic'>
\alt `func' <Basic'> `->' <Basic'>
\alt `rel' <Basic'> `->' <Basic'>
\alt `(' <Basic'> `,' <Basic'> <MoreBasics'> `)'
\alt <Basic'>

<Basic'> ::= <Id> | <Basic>

<Basic> ::= `int' | `char' | `double' | `bool'

\end{grammar}
\end{frame}

\begin{frame}[fragile]{La gramática, cont.}
\scriptsize
\begin{grammar}

<Invariant> ::= `\{inv' <BoolExp> `inv\}'

<ProcDecs> ::= <ProcDec> <ProcDecs>

<ProcDec> ::= `proc' <Id> `(' <ProcArgs> `)' <Precond> <Postcond>

<ProcArgs> ::= <ProcArg> `,' <ProcArgs> | $\lambda$

<ProcArg> ::= (`in'|`out'|`inout'|`ref') <Id> `:' <AbstType'>

<AbstType'> ::= <AbstType>
\alt <Id> <TypeVars>

<Precond> ::= `\{pre' <BoolExp> `pre\}'

<Postcond> ::= `\{post' <BoolExp> `post\}'

\end{grammar}
\end{frame}


\subsection{Tipos de datos estructurados}
\begin{frame}[fragile]{La gramática}
\scriptsize
\begin{grammar}

<DT> ::= `type' <Id> <Types> `begin' <DTBody> `end'


<Types> ::= `(' <Type> <MoreTypes> `)' | $\lambda$

<Type> ::= array of <Type>
\alt `*' <Type>
\alt <Basic>

<MoreTypes> ::= `,' <Type> <MoreTypes> | $\lambda$

<Body> ::= <VarDecs> <Repinv> <Coupinv> <ProcDefs>

<Repinv> ::= `\{repinv' <BoolExp> `repinv\}'

<Coupinv> ::= `\{coupinv' <BoolExp> `coupinv\}'

<ProcDefs> ::= <ProcDef> <ProcDefs>

<ProcDef> ::= \textit{(Definido por Araujo y Jiménez)}

\end{grammar}
\end{frame}


\subsection{Acoplamiento}
\begin{frame}{Verificación de precondiciones}
\begin{enumerate}
\item Se verifica la precondición del TD (concreto). Si falla, se da una
      \textbf{\textit{advertencia}} a tiempo de ejecución, ``No se cumplió la precondición'',
      y no se hacen más verificaciones dentro de esta llamada. Si se cumple,
      la verificación sigue.

\item Se utiliza el invariante de acoplamiento para generar las estructuras
      del TDA a partir de las del TD (concreto).

\item Se verifica la precondición del TDA con las estructuras generadas
      en el paso anterior. Si falla, se da un \textbf{\textit{error}} a tiempo de
      ejecución, ``Este procedimiento no implementa el abstracto''. Si se cumple,
      se ejecuta el cuerpo del método.
\end{enumerate}
\end{frame}


\begin{frame}{Verificación de poscondiciones}
\begin{enumerate}
\item Se utiliza el invariante de acoplamiento para generar las estructuras
      del TDA a partir de las del TD (concreto).

\item Se verifican tanto la poscondición del TD (concreto) como la del TDA,
      existiendo las siguientes posibilidades:

\vskip 0.5cm

\tiny\centering
\begin{tabular}{| l || c | c |}
  \hline
  Poscondición \ldots & abstracta se cumple & abstracta falla \\
  \hline \hline
  concreta se cumple & \parbox[t]{2cm}{Éxito} & \parbox[t]{3cm}{``Este procedimiento no implementa el abstracto, revisar acoplamiento''} \\
  \hline
  concreta falla & ``No cumple poscondición'' & ← y ↑ \\
  \hline
\end{tabular}

\end{enumerate}
\end{frame}


\subsection{Apuntadores}
\begin{frame}{Apuntadores}
\begin{itemize}
  \item Se desea un uso sencillo pero sin límites excesivos para los apuntadores
  \item Se considera apropiado manejar apuntadores exclusivamente a estructuras en el \textit{heap},
    manejados con palabras clave \texttt{new} y \texttt{free}.
  \item Interacción delicada con tipos de datos estructurados al momento de verificar aserciones.
\end{itemize}
\end{frame}