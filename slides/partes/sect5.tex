\section{Extras}

\defverbatim[colored]\pragma{
\begin{lstlisting}[language=graciela, style=code, escapechar=\~]
/*% LANGUAGE Trace %*/
\end{lstlisting}
}

\begin{frame}{Apoyo al programador}
\begin{itemize}
  \item Se simplificó el proceso de compilación, ahora basta con usar el comando \texttt{graciela}. Adicionalmente, se provee el comando \texttt{rungraciela}, que compila y ejecuta un programa en un solo paso.
  \item Se desarrollaron resaltadores sintácticos de Graciela para los editores Vim y Sublime Text. Adicionalmente, para este último editor se escribieron plantillas (\textit{Snippets}) de código.
  \item Por otro lado, se agregó al lenguaje la capacidad de imprimir trazas dentro de funciones con el pragma
  \pragma
  que habilita la familia de funciones \ingra{trace()}.
\end{itemize}
\end{frame}

