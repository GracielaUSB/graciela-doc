\section{ }

\begin{frame}{Apuntadores}
\framesubtitle{Recomendación de Araujo y Jiménez}
Esta extensión al lenguaje fue una recomendación explícita de Araujo y Jiménez
por su amplia utilidad para el curso <<Algoritmos y Estructuras II>>. Se permite
declarar variables de tipo apuntador a cualquier otro tipo del lenguaje fuera 
de la teoría de conjuntos.

\begin{itemize}
  \item{\makebox[4cm][l]{Declaración           } \ingra|var pt : T~~* | }
  \item{\makebox[4cm][l]{Reserva de memoria    } \ingra|new  (pt) | }
  \item{\makebox[4cm][l]{Liberación de memoria } \ingra|free (pt) | }
\end{itemize}
\end{frame}

\section{Apuntadores}

\begin{frame}{Paréntesis}

Asimismo, se extendió el reconocedor sintáctico de los tipos para aceptar el 
uso de paréntesis, a fin de permitir al programador distinguir entre, por 
ejemplo, un arreglo de apuntadores y un apuntador a un arreglo.

\begin{itemize}
  \item{\makebox[4cm][l]{Arreglo de apuntadores } \ingra|var ptarr : array [n] of (T~~*~~) | }
  \item{\makebox[4cm][l]{Apuntador a arreglo    } \ingra|var arrpt : (array [n] of T)* | }
\end{itemize}
\end{frame}

\begin{frame}{Debugging}

Para asegurarnos de que las funciones \texttt{new} y \texttt{free} funcionaban
de la manera esperada, hicimos uso de la herramienta \textit{Valgrind}\footfullcite{valgrind}. 
\begin{itemize}
  \item Toma un ejecutable.
  \item Lo enlaza con sus propias funciones de reserva y liberación de memoria.
  \item Elabora un reporte sobre posibles accesos indebidos y pérdida de memoria 
    dinámica.
\end{itemize}

Gracias a esto, descubrimos varias fugas de memoria y accesos indebidos, específicamente
en el uso de arreglos. La forma en que habíamos implementado inicialmente los arreglos conformantes presentaba fallos y estos fueron solventados exitosamente.
\end{frame}

\subsection*{Aserciones}
\begin{frame}{Lógica de separación}
Para poder expresar aserciones sobre estructuras que hacen uso de apuntadores, 
el Prof. Monascal nos orientó hacia la extensión de la Lógica de Hoare conocida 
como Lógica de Separación\footfullcite{separation-logic}.

\begin{itemize}
  \item No le agrega poder a la Lógica de Hoare.
  \item Pero aporta expresividad y permite enunciar propiedades complejas de 
  una manera más sucinta.
 \end{itemize}
\end{frame}

\begin{frame}{Lógica de separación, semántica}

\begin{description}
  \item [emp] Es la constante que representa un \textit{heap} vacío. 
  \item [$*$] Es la \textit{conjunción separadora}. $P*Q$ significa que P y Q son verdaderas en porciones disjuntas de la memoria.
  \item [$\mapsto$] Es el operador de \textit{deconstrucción de heap unitario}. $E \mapsto x$ significa que E apunta al fragmento de memoria $x$. Suele usarse de la forma $E \mapsto [l : x, r : y]$, equivalente a $(E \mapsto x) * (E + 1 \mapsto y)$
\end{description}

\end{frame}

\begin{frame}{Lógica de separación, cont.}
\framesubtitle{Ejemplo motivador}

\begin{align*}
  tree(E) \Longleftrightarrow\ &\boldsymbol{if}\ isatom?(E)\ \boldsymbol{then}\ emp \\
             &\boldsymbol{else}\ \exists xy.\ E\mapsto[l: x,\ r: y] * tree(x) * tree(y) \nonumber
\end{align*}

\end{frame}

\begin{frame}{Lógica de separación, cont}
Sin embargo,
\begin{itemize}
  \item Difícil implementación debido a complejas estructuras internas (simulación del heap).
  \item Nuevo sistema formal para estudiantes de cursos introductorios. 
\end{itemize}
\end{frame}
