\usepackage{listings}
\usepackage{etoolbox}
\usepackage{xcolor}
\usepackage{caption}
\usepackage{relsize}

\setmonofont[
  Contextuals={Alternate},
  BoldFont={Fira Code Bold},
  ItalicFont={Fira Sans Italic},
  BoldItalicFont={Fira Sans Bold Italic}
  ]{Fira Code}
\newfontfamily{\monofallbackfont}{DejaVu Sans Mono}
\DeclareTextFontCommand{\textmonofallback}{\monofallbackfont\textscale{0.95}}
\newcommand{\qop} {\color{cyan process}{\textmonofallback{\large ★}}}
\newcommand{\op}  {\color{cyan process}{\textmonofallback{⊕ }}}
\newcommand{\Lbag}{\textmonofallback{⟅}}
\newcommand{\Rbag}{\textmonofallback{⟆}}
\newcommand{\Lseq}{\textmonofallback{⟨}}
\newcommand{\Rseq}{\textmonofallback{⟩}}
\newcommand{\Union}{\textmonofallback{∪}}
\newcommand{\Land}{\textmonofallback{∧}}
\newcommand{\Lor}{\textmonofallback{∨}}
\newcommand{\Elem}{\textmonofallback{∈}}
\newcommand{\Notelem}{\textmonofallback{∉}}
\newcommand{\Forall}{\textmonofallback{∀}}
\newcommand{\Exist}{\textmonofallback{∃}}


% greens
\definecolor{caribbeangreen}{rgb}{0.0, 0.6, 0.45}

% blues
\definecolor{cerulean}{rgb}{0.0, 0.48, 0.65}
\definecolor{cyan process}{rgb}{0.0, 0.36, 0.46}

% reds
\definecolor{crimson}{rgb}{0.86, 0.08, 0.24}
\definecolor{darkcandyapplered}{rgb}{0.64, 0.0, 0.0}

\definecolor{goldensand}{rgb}{0.70,0.62,0.32}

% code and mathematics
\lstdefinelanguage{graciela}{ %
    keywordstyle=\bfseries\color{crimson},
    % list of keywords
    keywords=[1]{},
    morekeywords=[2]{
        proc, func, program, package, import, begin, end, abstract, type, enum,
        if, do, in, out, inout, fi, od, write,
        writeln, read, from, where, of, implements,
        set, rel, relation, multiset, sequence, function,
        var, const, array, int, boolean, float, char,
        new, free, abort, warn, skip, random
    },
    keywordstyle=[2]{\color{cerulean}\bfseries},
    % symbol keywords
    alsoletter={+/\\-*<>=!\{\}},
    morekeywords=[3]{
        +,  /\\, \\/, ∧, ∨, -, *, /, div, mod, \\, ->, →, <->,
        ↔, [, ], elem, notelem, ∈, ∉, union, intersect, ∪, ∩, \Union
        msum, ⊎, ++, ⧺, subset, ssubset, superset, ssuperset,
        ⊆, ⊂, ⊊, ⊇, ⊃, ⊋, ×, ÷, ^, sqrt, √, ⇒ ⇐, ≡, ≢,
        ==>, <==, ===, !==, ==, !=, ≠, <=, <, >, >=, ≤, ≥, !, ¬,
        forall, exist, notexist, max, min, product, sum, count,
        toFloat, toInt, toChar, toBoolean, toSequence, toSet, toMultiset
        ∀, ∃, ∄, ∑, ∏, \#
    },
    keywordstyle=[3]\color{cyan process},
    morekeywords=[4]{\{repinv, \{coupinv, \{pre, \{post, \{bound, \{inv, repinv\}, coupinv\}, pre\}, post\}, bound\}, inv\}},
    keywordstyle=[4]\color{darkcandyapplered},
    morekeywords=[5]{true, false, null},
    keywordstyle=[5]\color{caribbeangreen},
    numbers=none,
    stepnumber=1,
    showstringspaces=false,
    breaklines=false,
    comment=[l]{//},
    morecomment=[s]{/*}{*/},
    stringstyle=\color{goldensand}\ttfamily,
    morestring=[b]',
    morestring=[b]",
    sensitive=true,                 % keywords are case-sensitive
    morestring=[b]",                % strings are in double quotes
    morestring=[d]',                % characters are in single quotes
}

\newcommand{\inlinemath}[1]{{\small\bfseries\color{cyan process}\texttt{$#1$}}}
\newcommand{\inlinecode}[1]{{\small\bfseries\color{darkcandyapplered}\texttt{#1}}}
\lstdefinestyle{code}{%
    basicstyle=\ttfamily\scriptsize,
    commentstyle={\color{gray}},        % style for comments
    % captionpos=t,                       % caption position: top
    frame=none,                    % frame: [l]eft [r]ight [t]op [b]ottom
    rulecolor=\color{gray},
    columns=flexible,                   % flexible character width
    keepspaces=true,                    % keep all spaces
    % margins
    floatplacement=h,
    aboveskip=1cm, xleftmargin=1cm, xrightmargin=1cm,
    % % listings does not support UTF-8
    literate= {á}{{\'a}}1 {é}{{\'e}}1 {í}{{\'i}}1 {ó}{{\'o}}1 {ú}{{\'u}}1
              {Á}{{\'A}}1 {É}{{\'E}}1 {Í}{{\'I}}1 {Ó}{{\'O}}1 {Ú}{{\'U}}1
              {à}{{\`a}}1 {è}{{\`e}}1 {ì}{{\`i}}1 {ò}{{\`o}}1 {ù}{{\`u}}1
              {À}{{\`A}}1 {È}{{\'E}}1 {Ì}{{\`I}}1 {Ò}{{\`O}}1 {Ù}{{\`U}}1
              {ä}{{\"a}}1 {ë}{{\"e}}1 {ï}{{\"i}}1 {ö}{{\"o}}1 {ü}{{\"u}}1
              {Ä}{{\"A}}1 {Ë}{{\"E}}1 {Ï}{{\"I}}1 {Ö}{{\"O}}1 {Ü}{{\"U}}1
              {â}{{\^a}}1 {ê}{{\^e}}1 {î}{{\^i}}1 {ô}{{\^o}}1 {û}{{\^u}}1
              {Â}{{\^A}}1 {Ê}{{\^E}}1 {Î}{{\^I}}1 {Ô}{{\^O}}1 {Û}{{\^U}}1
              {œ}{{\oe}}1 {Œ}{{\OE}}1 {æ}{{\ae}}1 {Æ}{{\AE}}1 {ß}{{\ss}}1
              {ű}{{\H{u}}}1 {Ű}{{\H{U}}}1 {ő}{{\H{o}}}1 {Ő}{{\H{O}}}1
              {ç}{{\c c}}1 {Ç}{{\c C}}1 {ø}{{\o}}1 {å}{{\r a}}1 {Å}{{\r A}}1
              {€}{{\EUR}}1 {£}{{\pounds}}1 {¿}{{\textquestiondown}}1
              {¡}{{\textexclamdown}}1
}

\lstset{
    style=code,
    escapechar=\~,
}

\lstnewenvironment{gracielacode}[1][]{
    \linespread{1}
    \lstset{
        language=graciela,
        style=code,
        escapechar=\~,
        #1                  % settings for the lst environment
    }
  }{
  }

\lstnewenvironment{widegracielacode}[1][]{
    \linespread{1}
    \lstset{
        language=graciela,
        style=code,
        escapechar=\~,
        xleftmargin=0cm,
        xrightmargin=0cm,
        #1                  % settings for the lst environment
    }
  }{
  }

% http://tex.stackexchange.com/q/43526
% fix the apparently deliberate but undocumented behaviour of disabling escapes other than mathescape in TextStyle (used by \lstinline)
% there may be a good reason why this is disabled by default, so beware in case it causes any problems
\makeatletter
\patchcmd{\lsthk@TextStyle}{\let\lst@DefEsc\@empty}{}{}{\errmessage{failed to patch}}
\makeatother
\newcommand{\ingra}{\lstinline[language=graciela]}

\lstnewenvironment{haskellcode}[1][]{
    \linespread{1}
    \lstset{
        language=haskell,
        keywords={          % taken from wiki.haskell.org/Keywords
            as, case, class, data, default, deriving, do, else, family, forall,
            foreign, hiding, if, import, in, infix, infixl, infixr, instance,
            let, mdo, module, newtype, of, proc, qualified, rec, then, type,
            where
        },
        keywordstyle={\color{cerulean}\bfseries},
        keywordstyle={[2]\bfseries},
        style=code,
        #1                  % settings for the lst environment
    }
}{}

\renewcommand{\lstlistingname}{Fragmento de Código}

\let\oldtexttt\texttt
\DeclareTextFontCommand{\texttt}{\ttfamily\footnotesize}


\DeclareCaptionFont{white}{\color{white}}
\DeclareCaptionFormat{listing}
  {\centering #1#2#3}
\captionsetup[lstlisting]{format=listing}
