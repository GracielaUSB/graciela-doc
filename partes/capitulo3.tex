\chapter{Desarrollo}
\label{capitulo3}
\lhead{Capítulo 3. \emph{Desarrollo}}

El diseño y desarrollo de la extensión al compilador de Graciela fueron llevados
a cabo en tres etapas, entre marzo y noviembre del año 2016. En la primera
etapa,
 se estudió el estado del compilador elaborado por Araujo y Jiménez,
 se evaluaron las recomendaciones que sobre la semántica de este lenguaje hizo el
jurado de este primer proyecto,
 se revisó la bibliografía relacionada con el manejo de tipos definidos por el
usuario en el contexto de programación formal,
 se investigó sobre posibles estructuras de datos para implantar tipos que modelen la teoría de conjuntos,
 se especificó formalmente la sintaxis para las nuevas funcionalidades propuestas
y
 se extendieron los analizadores lexicográfico y sintáctico para concordar con
dicha especificación formal.

En la segunda etapa,
 se completó la verificación de tipos en presencia de tipos definidos por el
usuario y tipos que modelan la teoría de conjuntos,
 se extendió la biblioteca
externa de Graciela para soportar expresiones de tipos que modelan la teoría de
conjuntos,
 se inició la extensión al generador de código intermedio LLVM para producir las
instrucciones correspondientes a las nuevas funcionalidades y
 se escribió una colección de programas que ejercitan las capacidades del
lenguaje, tanto nuevas como originales, a fin de evaluar que el código generado
fuera correcto.

En la tercera etapa,
 se culminó la extensión al generador de código intermedio iniciada en la etapa
anterior,
 se extendió el manual de usuario para presentar las nuevas funcionalidades del
lenguaje,
 se investigaron formas para permitir que usuarios noveles instalen el compilador
sin mayor dificultad
 se estudiaron las herramientas necesarias para incorporar facilidades de
depuración (\emph{debugging}) y análisis de rendimiento (\emph{profiling}) al
compilador.

\section{Primera etapa}
\subsection{Revisión del proyecto de Araujo y Jiménez}
\blindtext[1]
% - Recomendaciones
% - Modularizar proyecto para permitir futuros desarrollos
% - Reestructuración de la gramática de los programas en Graciela

\subsection{Recuperación de errores en análisis sintáctico y Megaparsec}
\blindtext[1]
%     - Razones Megaparsec > Parsec
%         - Recuperación de errores -> `withRecovery`
%         - Monad Transformers
%         - Análisis lexicográfico más sencillo

\subsection{Consideraciones sobre arreglos}
\blindtext[1]
% - Arreglos multidimensionales en lugar de arreglos de arreglos
% - Verificar tamaños de arreglos

\subsection{Consideraciones sobre funciones y procedimientos}
\blindtext[1]
% - Cota
% - Cambios en proc (sin : antes de los parámetros)
%     - {pre} y {post} están juntas, {bound} opcional para procedimientos
%       recursivos
%     - se elminó el modo de parámetros "ref", equivalente a in-out -- MONASCAL?

\subsection{Consideraciones sobre cuantificaciones}
\blindtext[1]
%     - Posibilidad de usar colecciones como rango
%     - Separación del "rango" en "rango" y condiciones

\subsection{Teoría de conjuntos}
\blindtext[1]
% - Similares a cuantificaciones, aplican restricciones similares
% - Notación para relation y function:
%     : relation int <-> char, rel(<conjunto de pares>)
%     : function char -> float, func(<conjunto de pares>)

\subsection{Tipos definidos por el usuario}
\blindtext[1]
%  - breve discusión de Carroll Morgan
%  - LA GRAMÁTICA VA AL FINAL PERO IGUAL HABLAMOS SOBRE ELLA
% ¿Qué es un tipo de dato (abstracto/concreto)?
        % - Consideraciones
        % - Qué significa que un TDC implemente un TDA, relación de refinamiento
        % - Variables de tipo
        % - Invariantes de representación, acoplamiento
        % - Procedimientos y Funciones de/sobre un TAD

\section{Segunda etapa}
\subsection{Verificación de tipos}
\blindtext[1]
% - breve discusión de =:= y <> de tipos, también de funciones con argumentos
%   con variables de tipo


\subsection{Apuntadores}
\blindtext[1]
        % - malloc
        % - free
        % - Apuntadores y estructuras (malloc recibe tamaño en bytes, debe ser calculado: getElementPointer)

\subsection{Biblioteca externa para teoría de conjuntos}
\blindtext[1]
% - Backend en C++ (std)

\subsection{Generación de código intermedio}
\blindtext[1]
% TDDDPEU
% Cotas de funciones y procedimientos
% Cuantificaciones
% Conjuntos

\subsection{Colección de programas}
\blindtext[1]
% Referencia a los anexos

\section{Tercera etapa}
\subsection{Generación de código intermedio (cont.)}
\blindtext[1]
% - Variables de tipo
% - Invariantes de representación, acoplamiento
%     - Invariante de acoplamiento "compilable"
% - Procedimientos y Funciones de/sobre un TAD

\subsection{Facilidad de instalación para usuarios noveles}
\blindtext[1]

\subsection{Depuración y análisis de rendimiento}
\blindtext[1]
