\chapter*{Conclusiones y Recomendaciones}
\label{conclusiones}
\setchapter{\emph{Conclusiones y Recomendaciones}}

Durante el desarrollo de este Proyecto de Grado, se diseñó una importante
extensión al lenguaje de programación Graciela y se implementaron los cambios
correspondientes en su compilador. Con esta extensión, es posible utilizar
variables de tipo apuntador y definir estructuras con comportamientos abstractos
verificables a través del uso de invariantes lógicos, aprovechando expresiones
de la teoría de conjuntos. Así, se extiende la capacidad didáctica del lenguaje,
pues permite estudiar, con muy ligeros cambios respecto a los ejemplos
utilizados en la teoría, los temas propios del curso de Algoritmos y Estructuras
II de la Universidad Simón Bolívar. Adicionalmente, se mejoraron características
ya presentes en el lenguaje y en el compilador, como la manipulación de
arreglos, el uso de cotas para las subrutinas y la escritura de
cuantificaciones, siempre con la intención de apoyar las prácticas disciplinadas
de programación a través del uso de estrategias formales.

Esta extensión fue implementada en dos partes, \textit{Front-End} y
\textit{Back-End}, cada una empleando un conjunto particular de herramientas.
Para el \textit{Front-End}, la parte del compilador encargada de analizar el
código fuente escrito en Graciela y a partir de éste producir un Árbol
Sintáctico Abstracto, se utilizó principalmente la biblioteca de análisis
sintáctico \textit{Megaparsec}, así como un Monad en Haskell especializado para
este propósito. Respecto a la versión original del compilador de Graciela, ahora
se reconocen estructuras más diversas, como las definiciones de tipos de dato
abstractos o sus implementaciones, expresiones con apuntadores o expresiones de
la teoría de conjuntos, y en caso de errores de sintaxis o semántica, se proveen
mensajes de error más claros.

Para el \textit{Back-End}, se aprovechó la familia de software LLVM, a través de
las bibliotecas de Haskell \textit{llvm-general} y \textit{llvm-general-pure}.
Estas bibliotecas permiten emitir código intermedio de LLVM, que luego es
optimizado y convertido en código nativo de alta calidad a través del compilador
LLVM. Adicionalmente, se usaron los lenguajes C y C++, así como la librería
estándar de este último, para implementar las bibliotecas externas de funciones
para los programas escritos en Graciela.

Así, el producto final de este Proyecto de Grado es un lenguaje, junto con su
compilador, dirigido a la enseñanza de algoritmos a través de métodos formales y
especialmente diseñado para los cursos de Algoritmos y Estructuras I y II de la
Universidad Simón Bolívar, que incluye, adicionalmente a las características
presentes en su versión original:

\begin{itemize}
  \item Instalación y uso sencillos
  \item Procedimientos y funciones con cotas numéricas.
  \item Pase de parámetros constantes.
  \item Pase de arreglos como parámetros de modo \textit{In}, \textit{In-Out} y \textit{Out}.
  \item Expresiones de la teoría de conjuntos.
  \item Manipulación de apuntadores con las instrucciones \ingra{new} y \ingra{free}.
  \item Definición de Tipos de Dato Abstractos con sus invariantes y procedimientos abstractos.
  \item Implementación de Tipos de Dato Abstractos, con verificaciones para la relación de implementación.
\end{itemize}

Para futuros proyectos dirigidos a mejorar y ampliar el lenguaje Graciela, se
presentan las siguientes recomendaciones:

\begin{itemize}

  \item Agregar soporte al compilador en el sistema operativo Microsoft Windows.
  Para esto, debe estudiarse la manera más apropiada de empaquetar y distribuir
  el compilador para este sistema. Gracias a que el compilador GHC y las familia
  de herramientas LLVM cuentan con soporte para Windows, los cambios en el
  código fuente serán despreciables.

  \item Permitir la compilación de varios archivos a través de un nuevo par de
  palabras reservadas, \ingra{package} e \ingra{import}. Una de las
  características de Megaparsec es que el analizador sintáctico permite cambiar
  el archivo que se está analizando en un momento dado, y bastaría asociar la
  palabra \ingra{import} con un cambio como este.

  \item Proveer la capacidad de definición de tipos de datos enumerados,
  posiblemente integrándolos al sistema de operadores aritméticos y relacionales
  del lenguaje, y con la capacidad de ser usados en las construcciones que
  aceptarían otros tipos básicos, como expresiones de la teoría de conjuntos.
  Para esto, bastaría definir una palabra reservada, por ejemplo \ingra{enum},
  y almacenar los identificadores definidos para cada tipo en la tabla de símbolos,
  para finalmente utilizar enteros como la representación en código intermedio.

  \item Permitir la definición de Tipos Algebraicos Libres. Esto es un caso más
  general de la recomendación anterior, de modo que sólo tiene sentido
  implementar una o la otra. En este caso, se debe permitir que los valores del
  tipo sean funciones de aridad arbitraria, posiblemente recursivas y con
  variables de tipos.

  \item Considerando que la biblioteca externa para tipos de la teoría de
  conjuntos está escrita en C++, usando la librería estándar de este lenguaje,
  sería un ejercicio interesante de desarrollo de compiladores reescribir esta
  librería en Graciela. En definitiva, cada uno de los tipos provistos por esta
  biblioteca puede ser visto como un TDA. Para lograr esto, sería necesario
  modificar el generador de código del compilador para producir código
  reubicable, pero una vez hecho esto sólo faltaría definir los TDAs.

  \item Una de las limitaciones que se introdujo al lenguaje fue que las
  expresiones de la teoría de conjuntos sólo pueden ser de un nivel, es decir,
  no se permiten colecciones de colecciones. Esto no siempre es suficiente en el
  mundo de los algoritmos, por lo que sería valioso agregar expresiones
  arbitrariamente anidadas de estos tipos.

  \item Otra limitación en el estado actual del lenguaje está en las variables
  de tipos, ya que estas sólo pueden ser instanciadas en tipos no-básicos.
  Permitir TDAs que tomen otros TDAs en sus variables de tipos permitiría
  construcciones más interesantes, como listas de listas o pilas de
  diccionarios.

  \item Permitir la definición de subrutinas correcursivas. Actualmente, las
  subrutinas sólo pueden llamarse a sí mismas (si tienen cota) o a subrutinas
  definidas antes de sí mismas, por lo cual no es posible crear correcursiones.
  Para poder llamar subrutinas definidas más abajo de sí mismas, debe agregarse
  a la gramática la posibilidad de declarar subrutinas <<por adelantado>>, por
  ejemplo escribiendo la firma de la subrutina sola y posteriormente con su
  cuerpo. Para asegurar que los programas terminan, esto debe hacerse
  simultáneamente con un cambio sutil en el código generado para las subrutinas:
  ahora, cada una lleva registro no sólo de sus propias expresiones de cota,
  sino de las cotas todas las subrutinas que la llaman correcursivamente, y en
  cada iteración debe asegurarse que el conjunto de cotas disminuyó.

  \item Para poder hacer aseveraciones acerca de apuntadores con mayor
  comodidad, sería valioso agregar la posibilidad de escribir expresiones de la
  lógica de separación en las aserciones del lenguaje. Para permitir esto, debe
  estudiarse cuidadosamente la definición de los operadores de esta lógica y
  deben definirse las estructuras internas de los programas compilados que
  mantendrán la información sobre el estado del \textit{heap} para cada
  aserción.

\end{itemize}
