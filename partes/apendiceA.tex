\chapter{Tripleta de Hoare para llamadas a procedimientos con TDAs}
\label{derivhoare}
\setchapter{Apéndice A. \emph{Tripleta de Hoare para llamadas a procedimientos con TDAs}}

En la sección \ref{tda1} se mostró la siguiente tripleta de Hoare:

\begin{alignat}{1}%1
  \{ P[X,Y:=A,&B]\, \land \, Invs(A \cup B)\} \nonumber \\
  p\  (A,&B,C) \label{eqn:tdatriple2} \\ 
  \{ Q[X,Y,Z:=A_0,&B,C]\, \land \, Invs(B \cup C)\} \nonumber
\end{alignat}

Esta tripleta corresponde a las llamadas a procedimientos donde  posiblemente
algunos parámetros son de Tipos de Dato Abstractos, donde $p$ es un
procedimiento definido como $\textbf{proc}\ p (in\ X : \textrm{TX},
in\textrm{-}out\ Y : \textrm{TY}, out\ Z : \textrm{TZ})$, $P$ y $Q$ son,
respectivamente, la precondición y postcondición de este procedimiento, $X$, $Y$
y $Z$ son listas de parámetros formales que posiblemente incluyen variables de
Tipos de Dato Abstractos, $A$, $B$ y $C$ son listas de parámetros reales pasados
al procedimiento $p$ tal que todos sus tipos corresponden a los de $X$, $Y$ y
$Z$ respectivamente, $A_0$ es la lista con los valores iniciales de los valores
en $A$, e $Invs(X)$ es equivalente a la expresión 
$(\forall\, x\ |\ x \in X : Inv(x) )$, donde $Inv(x)$ es el predicado que 
condensa todos los invariantes del valor $x$ si $x$ es de un TDA.

Adicionalmente, y sin pérdida de generalidad, supondremos que $P$ es de la forma $X = A_0 \land P'$. Esto es posible dado que $A_0$ es una variable de 
especificación, y si $P$ no contiene esta igualdad, se puede suponer que es
una abreviación de $P \land X = A_0$.

La postcondición de esta tripleta es, simplemente, la postcondición del procedimiento $p$ ($Q$), sustituyendo las listas de parámetros formales $Y$, $Z$ y $X$ por $B$, $C$ y el valor inicial de $A$ ($A_0$), respectivamente, y la verificación de los invariantes de todas las variables de Tipos de Dato Abstractos que estuvieran en las listas $B$ y $C$. No se verifican los 
invariantes de las variables de Tipos de Dato Abstractos que estuvieran en 
$A_0$ puesto que la lista de parámetros formales $X$ es pasada en modo $in$. Formalmente, $Post := Q[X,Y,Z := A_0,B,C]\, \land \, Invs(B \cup C)$.

La precondición de esta tripleta, por otro lado, fue derivada usando la fórmula de $wp$ (\textit{weakest precondition} o precondición más débil) de \cite{flaviani}, según la cual $wp([P, Q], Post)$ es equivalente a $P (x,y) \land (\forall\, y'\,|\,Q(x, y') : Post (x, y'))$, como se explica a continuación.

Como se desea conocer la precondición más débil de la llamada $p\  (A,B,C)$, con especificación $[P, Q]$, dada la postcondición $Post$ explicada anteriormente, tenemos

\begin{alignat}{1}%2
wp(p\  (A,B,C), Post)
\end{alignat}

Por definición de la instrucción de llamada a procedimiento, según la cual se sustituyen los parámetros formales en modos $in$ e $in\textrm{-}out$ por los correspondientes parámetros formales, luego se ejecuta el
cuerpo del procedimiento y por último se sustituyen los parámetros
actuales en modos $in\textrm{-}out$ y $out$ por los correspondientes parámetros formales, esto es equivalente a

\begin{alignat}{1}%3
wp(X,Y := A,B; [P,Q]; B,C := Y,Z, Post)
\end{alignat}

Esto a su vez es equivalente a

\begin{alignat}{1}%4
wp(X,Y := A,B; [P,Q], Post[B,C := Y,Z] \land Invs(Y \cup Z))
\end{alignat}

Aplicando $wp$ de nuevo,

\begin{alignat}{1}%5
wp(X,Y := A,B, wp([P,Q], Post[B,C := Y,Z] \land Invs(Y \cup Z)))
\end{alignat}

Lo cual por la definición de $wp$ es equivalente a

\begin{alignat}{1}%6
wp(X,Y := A,B, P \land (\forall\, B',C'\ &|\ Q [Y,Z := B',C'] \land Invs (B' \cup C') \nonumber \\
                                       &: Post [B,C := B',C'] \land Invs (B' \cup C')))
\end{alignat}

Esto último puede simplificarse a

\begin{alignat}{1}%7
wp(X,Y := A,B, P \land (\forall\, B',C'\ &|\ Q [Y,Z := B',C'] \land Invs (B' \cup C') \nonumber \\
                                       &: Post [B,C := B',C']))
\end{alignat}

Aplicando la definición de $Post$,

\begin{alignat}{1}%8
wp(X,Y := A,B, P \land (\forall\, B',C'\ &|\ Q [Y,Z := B',C']\, \land Invs (B' \cup C')  \nonumber \\
                                       &:  Q [X,Y,Z := A_0, B',C']\, \land Invs (B' \cup C')
\end{alignat}

Que puede simplificarse a

\begin{alignat}{1}%9
wp(X,Y := A,B, P \land (\forall\, B',C'\ &|\ Q [Y,Z := B',C']\, \land\, Invs (B' \cup C')  \nonumber \\
                                       &:  Q [X,Y,Z := A_0, B',C'])
\end{alignat}

\newpage
Evaluando $wp$,

\begin{alignat}{1}%10
P[X,Y := A,B] \land Invs (A \cup B) \land (\forall\, B',C'\ &|\ Q [X,Y,Z := A,B',C']\, \land\, Invs (B' \cup C')  \nonumber \\
                                                          &:  Q [X,Y,Z := A_0, B',C'])
\end{alignat}

Usando la suposición de que $P \equiv X = A_0 \land P'$,

\begin{alignat}{3}%11
A = A_0 & \land P'[X,Y &&:= A,B] \land Invs (A \cup B) \nonumber \\
        & \land (\forall\, B',C'\ &&|\ Q [X,Y,Z := A,B',C']\, \land\, Invs (B' \cup C')  \nonumber \\
        &                         &&:  Q [X,Y,Z := A_0,B',C'])
\end{alignat}

Como $A = A_0$, podemos sustituir las ocurrencias de $A$ por $A_0$, quedando

\begin{alignat}{3}%11
A = A_0 & \land P'[X,Y &&:= A,B] \land Invs (A \cup B) \nonumber \\
        & \land (\forall\, B',C'\ &&|\ Q [X,Y,Z := A_0,B',C']\, \land\, Invs (B' \cup C')  \nonumber \\
        &                         &&:  Q [X,Y,Z := A_0,B',C'])
\end{alignat}

Como el rango de la cuantificación universal implica al cuerpo, ésta es cierta,
quedando simplemente

\begin{alignat}{1}%13
A = A_0 \land P'[X,Y := A,B] \land Invs (A \cup B)
\end{alignat}

Que es equivalente a

\begin{alignat}{1}%14
(X = A_0 \land P')[X,Y := A,B] \land Invs (A \cup B)
\end{alignat}

\newpage
Usando nuevamente la suposición de que $P \equiv X = A_0 \land P'$, tenemos

\begin{alignat}{1}%15
P[X,Y := A,B] \land Invs (A \cup B)
\end{alignat}

Esta última expresión es la precondición que se mostró en la tripleta.
