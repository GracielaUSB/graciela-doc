\chapter{Marco Tecnológico}
\label{capitulo2}
\setchapter{Capítulo 2. \emph{Marco Tecnológico}}


%%%%%%%%%%%%%%%%%%%%%%%%%%%%%%%%%%%%%%%%%%%%%%%%%%%%%%%%%%%%%%%%%%%%%%%%%%%%%%%%
%% LENGUAJES DE PROGRAMACIÓN %%%%%%%%%%%%%%%%%%%%%%%%%%%%%%%%%%%%%%%%%%%%%%%%%%%
%%%%%%%%%%%%%%%%%%%%%%%%%%%%%%%%%%%%%%%%%%%%%%%%%%%%%%%%%%%%%%%%%%%%%%%%%%%%%%%%

\section{Lenguajes de Programación}

Para la implementación de los distintos componentes del presente proyecto de
grado, se utilizaron los lenguajes Haskell, C y C++. El primero se usó para
escribir las distintas fases del compilador, mientras que los otros dos se
utilizaron para desarrollar las bibliotecas externas para los programas
compilados.

\subsection{Haskell}
Haskell~\cite{haskell} es un lenguaje de programación funcional puro compilado, de propósito
general, que se basa en el \textit{cálculo lambda}. Al ser un lenguaje
funcional puro, todo programa se escribe haciendo uso de expresiones
matemáticas puras y se asegura que para todo valor que se proporcione como
entrada, las funciones llamadas dentro de la ejecución del mismo, devolverán
siempre el mismo resultado. Esto se debe a la carencia de variables mutables y
efectos de borde, lo que permite a su vez, generar código ensamblador mas
astuto con respecto a los lenguajes imperativos.

Entre sus principales características está la de un sistema de tipos estático y
sin conversiones explícitas, es decir, para todas las estructuras declaradas en
el programa, se verifica su tipo a tiempo de compilación y toda conversión entre
tipo debe colocarse de manera explícita. Por otro lado, La evaluación de
expresiones es de forma perezosa, dando como consecuencia que ninguna expresión
se evalúe hasta el momento en el que se usa y la posibilidad de poder crear
estructuras infinitas, las cuales son construidas parcialmente mientras van
siendo utilizadas.

Escribir un programa haciendo uso exclusivamente de expresiones matemáticas
puras, trae consigo limitaciones cuando se requiere realizar operaciones de
entrada y salida, o cualquier otra operación que implique efectos de bordes.
Este tipo de cómputo se puede lograr dentro de un lenguaje funcional haciendo
uso de los \textit{monads}, estructuras que permiten representar cálculos una
como secuencia de operaciones y que proporcionan contexto a las expresiones
puras. Estos contextos monádicos permiten realizar cómputos de entrada y
salida, manejo de excepciones, entre otros, sin dejar de lado la pureza
funcional.

Uno de los factores que se debe tomar en cuenta cuando se elige un lenguaje de
programación, es la cantidad y calidad de las bibliotecas disponibles para el
mismo. Haskell, posee una amplia gama de bibliotecas de mucha utilidad, entre la
cuales destacan las bibliotecas que permiten depurar y perfilar programas,
manejar y representar estructuras complejas, y en el caso del presente
proyecto, bibliotecas que proporcionan facilidades para construir reconocedores
recursivos descendentes y para generación de código intermedio LLVM, las
cuales serán mencionadas en detalles en las siguientes secciones.


\subsection{C, C++}

\textit{C}~\cite{c-lang} y \textit{C++}~\cite{cpp-lang} son lenguajes
imperativos, y orientado a objetos en el caso de \textit{C++}. Estos dos
lenguajes poseen un sistema de tipo estático y con conversiones implícitas, pero
ofrecen ventajas para realizar operaciones a muy bajo nivel, como el manejo y
acceso manual de la memoria.

Ambos lenguajes poseen mecanismo para declarar estructuras de datos básicos,
funciones y tipos enumerados, sin embargo, con \textit{C++}, al ser un lenguaje
orientado a objetos, entran conceptos como clases, herencia de clases y
encapsulamiento. También, a diferencia del lenguaje \textit{C}, \textit{C++}
posee una biblioteca estándar mas extensa, en la cual se ofrecen estructuras de
datos mas complejas como mapas, conjuntos o vectores, y algoritmos de
ordenamiento o búsqueda sobre dichas estructuras.

%%%%%%%%%%%%%%%%%%%%%%%%%%%%%%%%%%%%%%%%%%%%%%%%%%%%%%%%%%%%%%%%%%%%%%%%%%%%%%%%
%% HERRAMIENTAS PARA ANÁLISIS LÉXICO Y SINTÁCTICO %%%%%%%%%%%%%%%%%%%%%%%%%%%%%%
%%%%%%%%%%%%%%%%%%%%%%%%%%%%%%%%%%%%%%%%%%%%%%%%%%%%%%%%%%%%%%%%%%%%%%%%%%%%%%%%
\section{Herramientas para Análisis Léxico y Sintáctico}

\subsection{Parsec}

\textit{Parsec} es una biblioteca destinada a facilitar herramientas para la
construcción de analizadores sintácticos recursivos descendentes~\cite{parsec}.
Esta biblioteca provee estructuras monádica con las que se pueden generar
reconocedores sintácticos básicos y combinarlos para formar reconocedores cada
vez más complejos.  Esta biblioteca provee combinadores como \texttt{many} y
\texttt{<|>} los cuales permiten, respectivamente, ejecutar cero o mas veces un
reconocedor, o ejecutar un segundo reconocedor en caso de que el primero falle.

Si bien esta biblioteca no ofrece soporte para recuperación de errores durante
el análisis sintáctico, éste puede implementarse haciendo uso de la función
\texttt{try}, la cual restablece el estado original del flujo de entrada en caso
de fallar el reconocedor pasado como argumento. Esto debe hacerse de manera
cuidadosa, puesto que, si el reconocedor en cuestión es complejo, la información
analizada antes del punto donde ocurrió el error se pierde~\cite{parsectry}. Lo
recomendable es usar un conjunto pequeño de lexemas, conocido en inglés como
\textit{lookahead}, para decidir el camino que debe tomar el analizador
sintáctico en un punto donde existen múltiples opciones.

\subsection{Megaparsec}

La biblioteca \textit{Megaparsec}~\cite{megaparsec} es un bifurcación
(\emph{fork}, en inglés) de la biblioteca \textit{Parsec}, la cual además de
ofrecer la mismas funcionalidades que ésta ultima, agrega otras funcionalidades
de especial importancia para la elaboración de un analizador sintáctico, en
particular el soporte para la recuperación de errores, a través del combinador
\texttt{withRecovery}. Adicionalmente, en esta biblioteca, el \textit{Monad}
para el análisis sintáctico forma parte de diversas instancias de Haskell, de
modo que se pueden usar transformadores de \textit{Monads} sobre el analizador
sintáctico, lo cual permite manejar estados específicos para cada parte del
lenguaje a analizar con facilidad.


%%%%%%%%%%%%%%%%%%%%%%%%%%%%%%%%%%%%%%%%%%%%%%%%%%%%%%%%%%%%%%%%%%%%%%%%%%%%%%%%
%% HERRAMIENTAS PARA LA GENERACIÓN DE CÓDIGO INTERMEDIO %%%%%%%%%%%%%%%%%%%%%%%%
%%%%%%%%%%%%%%%%%%%%%%%%%%%%%%%%%%%%%%%%%%%%%%%%%%%%%%%%%%%%%%%%%%%%%%%%%%%%%%%%
\section{Herramientas para la Generación de Código Intermedio}

\subsection{LLVM}

El proyecto LLVM (originalmente las siglas de \textit{Low Level Virtual
Machine}, o <<Máquina Virtual de Bajo Nivel>>, pero ahora se trata de un nombre
propio) es una <<colección modular y reutilizable de tecnologías de compiladores
y cadenas de herramientas>>. Estas herramientas pueden ser usadas según sea
conveniente para construir y desarrollar compiladores de manera sencilla,
aprovechando una infraestructura sólida, con la capacidad de generar código
nativo para una gran variedad de plataformas. La manera más común de usar LLVM
para escribir un compilador es escribiendo el \textit{Front-End} para el
lenguaje que se desea compilar de modo que genere código intermedio
(\textit{IR}) de LLVM. Luego de esto, basta usar el \textit{Back-End} de LLVM
para convertir esta representación intermedia en código nativo para la
plataforma destino deseada.

Adicionalmente, el \textit{Back-End} de LLVM puede aplicar una gran variedad de
optimizaciones al código intermedio que recibe, mientras que las tecnologías de
cadenas de herramientas permiten desarrollar depuradores y perfiladores para los
lenguajes que aprovechan su \textit{Back-End}.

\subsection{llvm-general y llvm-general-pure}

Las bibliotecas \texttt{llvm-general} y
\texttt{llvm-general-pure}~\cite{llvm-general, llvm-general-pure} para el
lenguaje Haskell ofrecen una interfaz muy sencilla y completa con las
bibliotecas externas de LLVM para la generación de código intermedio. Por un
lado, \textit{llvm-general-pure} brinda una serie de estructuras de datos que se
corresponden con las definiciones, instrucciones y expresiones del código
intermedio de LLVM. A su vez, \textit{llvm-general} da las facilidades para
poder traducir, haciendo uso de las estructuras encontradas en
\textit{llvm-general-pure}, el árbol sintáctico generado por el analizador
sintáctico, a un nuevo árbol sintáctico a partir del cual posteriormente se
genera automáticamente un archivo con la representación intermedia de LLVM del
programa.

%%%%%%%%%%%%%%%%%%%%%%%%%%%%%%%%%%%%%%%%%%%%%%%%%%%%%%%%%%%%%%%%%%%%%%%%%%%%%%%%
%% HERRAMIENTAS PARA LA CONSTRUCCIÓN Y EJECUCIÓN DE CASOS DE PRUEBA %%%%%%%%%%%%
%%%%%%%%%%%%%%%%%%%%%%%%%%%%%%%%%%%%%%%%%%%%%%%%%%%%%%%%%%%%%%%%%%%%%%%%%%%%%%%%
\section{Herramientas para la Construcción y Ejecución de Casos de Prueba}

\subsection{HUnit}

El sistema de pruebas unitarias HUnit~\cite{hunit}, desarrollado en Haskell, 
permite ejecutar de manera sencilla y sistemática conjuntos específicos de 
pruebas unitarias para un programa, comparando los resultados con valores 
esperados de manera automática, con la intención de orientar el desarrollo 
hacia cumplir dichas pruebas. Como funciona dentro del Monad \texttt{IO}, es 
posible especificar pruebas que interactúan con el exterior, lo cual le brinda 
una gran flexibilidad. En este proyecto, esta herramienta permitió la escritura 
de una colección de programas de prueba que son compilados cada vez que se 
desea, con la finalidad de comparar su salida real (\texttt{stdout} y 
\texttt{stderr}) con la esperada para una entrada en particular 
(\texttt{stdin}).

\subsection{QuickCheck}

La biblioteca QuickCheck~\cite{quickcheck}, desarrollada originalmente en
Haskell y posteriormente imitada en otros lenguajes, ofrece una colección de
combinadores para verificar propiedades lógicas de manera aleatoria. Esto
significa que dada una propiedad parametrizada sobre un conjunto de variables,
QuickCheck genera variables de los tipos apropiados, los usa como argumentos de
la propiedad, y verifica si ésta se cumple. 

Para que QuickCheck sea capaz de generar valores aleatorios de un tipo de dato,
es necesario escribir la instancia correspondiente de la clase
\texttt{Arbitrary}. Como su nombre sugiere, los tipos que pertenecen a esta
clase pueden ser generados de manera arbitraria, siguiendo las instrucciones
dadas en la función \texttt{arbitrary}. QuickCheck incluye las instancias
\texttt{Arbitrary} de gran parte de los tipos básicos de Haskell, así como de
algunas bibliotecas populares del lenguaje, y adicionalmente ofrece múltiples
combinadores para definir nuevas instancias de esta clase, como \texttt{choose},
para generar un valor dentro de un rango, y \texttt{frequency}, para generar
valores a partir de una distribución ponderada de generadores.

En este proyecto, esta biblioteca permitió generar entradas aleatorias para
comparar la salida emitida por programas escritos en Graciela, con la producida
por una simulación escrita en Haskell. Esta técnica resulta muy superior a la
de HUnit, puesto que es posible hacer miles de pruebas sin escribir una sola.

%%%%%%%%%%%%%%%%%%%%%%%%%%%%%%%%%%%%%%%%%%%%%%%%%%%%%%%%%%%%%%%%%%%%%%%%%%%%%%%%
%% HERRAMIENTAS PARA LA DEPURACIÓN DEL MANEJO DE MEMORIA %%%%%%%%%%%%%%%%%%%%%%%
%%%%%%%%%%%%%%%%%%%%%%%%%%%%%%%%%%%%%%%%%%%%%%%%%%%%%%%%%%%%%%%%%%%%%%%%%%%%%%%%
\section{Herramientas para la Depuración del Manejo de Memoria}

\textit{Valgrind}~\cite{valgrind} es una herramienta de software que ayuda a la
depuración y detección errores relacionados a una mala gestión de la memoria de
un proceso, como accesos indebidos o la fuga de memoria dinámica. También
permite realizar un perfilaje detallado de la ejecución de un programa y
detectar errores en procesos multihilo.

En este proyecto, se utilizó \textit{Valgrind} con la intención de detectar
posibles fugas de memoria en el código de las bibliotecas externas de Graciela,
particularmente en el caso de las estructuras de tipos de la teoría de
conjuntos, y en el código generado para las nuevas instrucciones \ingra{new} y
\ingra{free} de apuntadores. Adicionalmente, se usó para revisar los accesos a
memoria relacionados con los arreglos, debido a la forma en la que fueron
implementados.

%%%%%%%%%%%%%%%%%%%%%%%%%%%%%%%%%%%%%%%%%%%%%%%%%%%%%%%%%%%%%%%%%%%%%%%%%%%%%%%%
%% AMBIENTES DE PROGRAMACIÓN %%%%%%%%%%%%%%%%%%%%%%%%%%%%%%%%%%%%%%%%%%%%%%%%%%%
%%%%%%%%%%%%%%%%%%%%%%%%%%%%%%%%%%%%%%%%%%%%%%%%%%%%%%%%%%%%%%%%%%%%%%%%%%%%%%%%
\section{Ambientes de programación}

Como objetivo adicional se estudió la posibilidad de incorporar herramientas
para mejorar la experiencia del programador en Graciela. Específicamente, se
indagó acerca de los editores de texto para programación más comúnmente usados
entre los estudiantes de la carrera de Ingeniería de la Computación y sobre cómo
agregarles características como un resaltador sintáctico y la inserción
automática de plantillas de código.

\begin{description}[leftmargin=!,labelwidth=\widthof{\bfseries Sublime Text}]

  \item [Vi / Vim] está presente en todos los sistemas operativos derivados de
  Unix. Permite mover, copiar, eliminar o insertar caracteres con mucha
  versatilidad. Tiene como ventaja su simplicidad y contar con una importante
  cantidad de paquetes que permiten personalizarlo a gusto del usuario. Es
  gratuito y de código abierto.

  \item [Sublime Text] es un editor de texto de ambiente gráfico escrito en el
  lenguaje C++ y con soporte para \textit{plug-ins} escritos en el lenguaje
  Python y distribuidos a través del manejador de paquetes diseñado
  exclusivamente para este editor \textit{Package Control}. Su código es cerrado
  y se debe adquirir una licencia para su uso prolongado, pero se ofrece una
  versión de prueba gratuita.

  \item [Atom] es un editor de texto escrito en el lenguaje JavaScript. El
  editor puede personalizarse a gusto del usuario pero tiene la desventaja de
  ser mas lento en comparación con otros editores compilados como Vim o Sublime
  Text. Incluye un manejador de paquetes similar al de Sublime Text. Es gratuito
  y de código abierto.

  \item [Gedit] viene con el entorno de escritorio GNOME y también puede ser
  personalizado por el usuario. Es gratuito y de código abierto.

\end{description}

%%%%%%%%%%%%%%%%%%%%%%%%%%%%%%%%%%%%%%%%%%%%%%%%%%%%%%%%%%%%%%%%%%%%%%%%%%%%%%%%
%% DISTRIBUCIÓN DE SOFTWARE %%%%%%%%%%%%%%%%%%%%%%%%%%%%%%%%%%%%%%%%%%%%%%%%%%%%
%%%%%%%%%%%%%%%%%%%%%%%%%%%%%%%%%%%%%%%%%%%%%%%%%%%%%%%%%%%%%%%%%%%%%%%%%%%%%%%%
\section{Distribución de software}

Para facilitar la distribución e instalación de los componentes de Graciela, se
estudiaron los mecanismos de distribución de software más comunes para las
plataformas en las cuales se ofrece el compilador, con la intención de que sea
sencillo agregar el compilador a sus sistemas, y mantenerlo actualizado con la
última versión disponible.

\begin{description}[leftmargin=!,labelwidth=\widthof{\bfseries Homebrew}]

\item [APT]~\cite{apt} (Advanced Package Tool, <<Herramienta Avanzada de
Paquetes>>) es una herramienta de software propia de Debian y distribuciones de
Linux derivadas de Debian, que administra la instalación y remoción de software
en estos sistemas operativos. Se decidió utilizar APT debido, precisamente, a
que está incluida en una gran cantidad de sistemas operativos, y a su facilidad
de uso.

\item [Homebrew]~\cite{brew} es un manejador de paquetes para el sistema
operativo macOS. Entre las razones por la cual se eligió este manejador de
paquetes, fue su simplicidad de uso, tanto para quien desea instalar una pieza
de software, como para quien crea y distribuye el paquete, comparación de las
demás opciones disponibles.

\end{description}
