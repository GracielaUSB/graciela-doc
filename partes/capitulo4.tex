\chapter{Resultados}
\label{capitulo4}
\setchapter{Capítulo 4. \emph{Resultados}}


El resultado del presente Proyecto de Grado es doble, pues se extendió el
lenguaje Graciela para incluir manejo de apuntadores, expresiones de la teoría
de conjuntos y tipos de dato definidos por el programador, y se dotó al
compilador \texttt{graciela} de la capacidad de analizar y generar código para
los usos de estas nuevas características del lenguaje. Adicionalmente, se
mejoraron algunas características ya presentes en el lenguaje y en el compilador
para mejorar la experiencia del programador novel.

%%%%%%%%%%%%%%%%%%%%%%%%%%%%%%%%%%%%%%%%%%%%%%%%%%%%%%%%%%%%%%%%%%%%%%%%%%%%%%%%
%% GRAMÁTICA %%%%%%%%%%%%%%%%%%%%%%%%%%%%%%%%%%%%%%%%%%%%%%%%%%%%%%%%%%%%%%%%%%%
%%%%%%%%%%%%%%%%%%%%%%%%%%%%%%%%%%%%%%%%%%%%%%%%%%%%%%%%%%%%%%%%%%%%%%%%%%%%%%%%
\section{Cambios en la gramática}

Para dar más claridad a los programas escritos en Graciela, se decidió que el
bloque principal debe estar precedido por la palabra clave \ingra{main}. Por
otro lado, en la definición de procedimientos y funciones, se movió la
poscondición junto con la precondición, de modo que aparecen antes del cuerpo de
la subrutina, eliminándose así la necesidad de usar las palabras clave
\ingra{begin} y \ingra{end} para delimitar el cuerpo de la subrutina. En las
subrutinas, también se eliminó el  símbolo \ingra{:} entre el identificador de
la subrutina y el conjunto de parámetros delimitado por paréntesis, por carecer
de valor semántico, pues no se trata de una declaración de tipos para las
subrutinas. Adicionalmente, se simplificó el símbolo utilizado para las
aserciones que se utilizan dentro de bloques de código imperativo, pasando de
\ingra|{a ~\textit{exp}~ a}| a \ingra|{~\textit{exp}~}|.

%%%%%%%%%%%%%%%%%%%%%%%%%%%%%%%%%%%%%%%%%%%%%%%%%%%%%%%%%%%%%%%%%%%%%%%%%%%%%%%%
%% RECUPERACIÓN DE ERRORES %%%%%%%%%%%%%%%%%%%%%%%%%%%%%%%%%%%%%%%%%%%%%%%%%%%%%
%%%%%%%%%%%%%%%%%%%%%%%%%%%%%%%%%%%%%%%%%%%%%%%%%%%%%%%%%%%%%%%%%%%%%%%%%%%%%%%%
\section{Recuperación de errores durante el análisis sintáctico}

El compilador \texttt{graciela} cuenta ahora con una mejor capacidad para
recuperarse de errores durante el análisis sintáctico de un archivo, y ofrece
mensajes de error más fáciles de entender en gran parte de los casos.
\todo{ejemplos de esto}

%%%%%%%%%%%%%%%%%%%%%%%%%%%%%%%%%%%%%%%%%%%%%%%%%%%%%%%%%%%%%%%%%%%%%%%%%%%%%%%%
%% ARREGLOS %%%%%%%%%%%%%%%%%%%%%%%%%%%%%%%%%%%%%%%%%%%%%%%%%%%%%%%%%%%%%%%%%%%%
%%%%%%%%%%%%%%%%%%%%%%%%%%%%%%%%%%%%%%%%%%%%%%%%%%%%%%%%%%%%%%%%%%%%%%%%%%%%%%%%
\section{Arreglos}

Para mejorar la experiencia de los programadores del lenguaje Graciela, se
agregaron al lenguaje verificaciones para cada acceso a un arreglo, de modo que
intentar usar posiciones más allá de los límites de un arreglo resulta en un
error a tiempo de ejecución, independientemente de si se intentaba leer del
arreglo o escribir a él. Dado que el uso equivocado de índices sobre arreglos
suele ser una fuente de defectos en el código, esta nueva característica puede
ser de gran utilidad para los estudiantes de los cursos introductorios de
algoritmos.

Por otro lado, se decidió ofrecer arreglos multidimensionales como parte del
lenguaje. Anteriormente, era posible utilizar arreglos multidimensionales
declarándolos como arreglos de arreglos de algún tipo, y para utilizar sus
elementos se hacía uso de tantos subíndices (\ingra{[i]}) como dimensiones
tuviera el arreglo. Ahora, son declarados como arreglos de varias dimensiones,
especificando entre corchetes sus tamaños en cada dimensión, separados por
comas, como en el fragmento de código \ref{lst:arrdec}.

\begin{gracielacode}[caption=Declaración de variable de tipo arreglo, label=lst:arrdec]
var unArreglo : array [x, y, z] of int
\end{gracielacode}

Similarmente, para hacer uso de sus elementos, se usan tantos subíndices como
dimensiones tuviera el arreglo de interés, pero entre un mismo par de corchetes,
como en el fragmento de código \ref{lst:arridx}.

\begin{gracielacode}[caption=Acceso a variable de tipo arreglo, label=lst:arridx]
writeln (unArreglo[1, 2, 3])
\end{gracielacode}

Es importante destacar que si se usa una secuencia de subíndices de más o menos
dimensiones que las que tiene el arreglo que se desea usar, se da un error
bastante claro a momento de compilación, especificando las dimensiones del
arreglo y de la secuencia de subíndices. Esto puede compararse con la
implementación original, en la que el error emitido mencionaría, o bien un
intento de subindizar una variable que no es un arreglo, o bien un intento de
usar un arreglo en donde se esperaba un valor de un tipo básico; estos errores,
a pesar de ser ciertos, no indican la verdadera razón del problema, la falta de
concordancia entre las dimensiones del arreglo y las de la subindización
utilizada.

%%%%%%%%%%%%%%%%%%%%%%%%%%%%%%%%%%%%%%%%%%%%%%%%%%%%%%%%%%%%%%%%%%%%%%%%%%%%%%%%
%% SUBRUTINAS %%%%%%%%%%%%%%%%%%%%%%%%%%%%%%%%%%%%%%%%%%%%%%%%%%%%%%%%%%%%%%%%%%
%%%%%%%%%%%%%%%%%%%%%%%%%%%%%%%%%%%%%%%%%%%%%%%%%%%%%%%%%%%%%%%%%%%%%%%%%%%%%%%%
\section{Subrutinas}

Anteriormente, se permitía que procedimientos y funciones fueran recursivos sin
importar si esta recursión era acotada. De hecho, no existía en la gramática del
lenguaje la posibilidad de dar una cota a procedimientos o a funciones. Ahora,
para poder hacer llamadas recursivas de una subrutina, es necesario declarar
para ella una expresión de cota, la cual, de manera equivalente a la cota de la
instrucción \ingra{do}, será evaluada cada vez que sea llamada la subrutina, y
en caso de no descender, o de volverse negativa, se dará el correspondiente
error a tiempo de ejecución. Si se define una cota para una subrutina que no es
recursiva, esta símplemente es ignorada.

%%%%%%%%%%%%%%%%%%%%%%%%%%%%%%%%%%%%%%%%%%%%%%%%%%%%%%%%%%%%%%%%%%%%%%%%%%%%%%%%
%% CUANTIFICACIONES %%%%%%%%%%%%%%%%%%%%%%%%%%%%%%%%%%%%%%%%%%%%%%%%%%%%%%%%%%%%
%%%%%%%%%%%%%%%%%%%%%%%%%%%%%%%%%%%%%%%%%%%%%%%%%%%%%%%%%%%%%%%%%%%%%%%%%%%%%%%%
\section{Cuantificaciones}

Las cuantificaciones ya formaban parte del lenguaje en su versión original, pero
se les hicieron algunas modificaciones. En primer lugar, la variable auxiliar
ahora puede ser de tipo \ingra{float} o de una variable de tipo, en adición a
los tipos permitidos anteriormente,  \ingra{int}, \ingra{char} o
\ingra{boolean}. Se permiten variables de tipo \ingra{float} gracias a la
adición de los tipos colección, explicados más adelante, ahora es posible
definir iteraciones sobre números de punto flotante, usando el operador de
pertenencia (\Elem) sobre una colección, mientras que se permiten las variables
de tipo debido a que éstas siempre serán instanciadas en alguno de los
tipos \ingra{int}, \ingra{char}, \ingra{boolean} o \ingra{float}).

En segundo lugar, ahora el compilador admite expresiones booleanas más generales
en el rango de una cuantificación. La versión original del lenguaje permitía
expresiones arbitrarias en el rango, pero el compilador correspondiente sólo
admitía una conjunción de dos operaciones relacionales, y fallaba si se usaba
otra forma de expresión. Ahora, se aceptan expresiones lógicas en Forma Normal
Conjuntiva, en las cuales se cumpla alguna de las siguientes condiciones:

\begin{description}%[leftmargin=!,labelwidth=\widthof{\bfseries Rango de pertenencia}]

  \item [Rango numérico] La variable auxiliar aparece, sola, en al menos dos
  desigualdades (\ingra{<}, \ingra{<=}, \ingra{>} o \ingra{>=}), como el
  operando menor y mayor en al menos una desigualdad, respectivamente, y
  adicionalmente el tipo de la variable auxiliar es \ingra{int} o \ingra{char}.
  En este caso, se forma un rango númerico entre el máximo de los límites
  inferiores y el mínimo de los límites superiores definidos por las
  desigualdades. Puede verse un ejemplo de esto en el fragmento de código
  \ref{lst:numrange}. En ese caso, el rango resultante es el comprendido entre $0$ y $100$,
  incluyendo  el límite inferior pero no el superior.

\begin{gracielacode}[caption=Cuantificación con rango numérico, label=lst:numrange]
(% ~\qop~ i : int | 0 <= i /\ i < 100 | ... %)
\end{gracielacode}

  \item [Rango de pertenencia] La variable auxiliar aparece del lado izquierdo
  de un operador de pertenencia (\Elem), con una colección en el lado derecho.
  En este caso, se forma un rango de pertenenecia sobre la colección en
  cuestión. Puede verse un ejemplo de esto en el fragmento de código
  \ref{lst:setrange}.  En ese caso, el rango resultante es el conjunto \ingra{s}.

\begin{gracielacode}[caption=Cuantificación con rango de pertenencia, label=lst:setrange]
var s : set of A
...
(% ~\qop~ i : A | i ~\Elem~ s | ... %)
\end{gracielacode}

  \item [Rango puntual] La variable auxiliar aparece, sola, en cualquier lado de
  un operador de igualdad (\ingra{==}). En este caso, se forma un rango puntual
  correspondiente a la expresión del otro lado del operador. Puede verse un
  ejemplo de esto en el fragmento de código \ref{lst:pointrange}. En ese caso,
  el rango resultante es el valor de la expresión
  \lstinline[language=graciela]{a ~\op~b ~\op~c}.

\begin{gracielacode}[caption=Cuantificación con rango puntual, label=lst:pointrange]
var a, b, c : int
...
(% ~\qop~ i : int | i == a ~\op~b ~\op~c | ... %)
\end{gracielacode}

  \item [Rango vacío] La expresión puede ser evaluada como la constante
  \ingra{false} a tiempo de compilación, resultando un rango vacío. Puede verse
  un ejemplo de esto en el fragmento de código \ref{lst:emptyrange}. En ese
  caso, el rango resultante es el rango vacío, con todas las consecuencias que
  pueda tener esto para el valor de la cuantificación y la ejecución del
  programa.

\begin{gracielacode}[caption=Cuantificación con rango puntual, label=lst:emptyrange]
(% ~\qop~ i : A | false | ... %)
\end{gracielacode}

\end{description}

De cumplirse más de una de las condiciones simultáneamente, se utiliza como
rango de iteración el primero de los rangos formados según la siguiente lista, y
la pertenencia en el segundo se evaluará para cada elemento de la iteración.

\begin{enumerate}
  \item Rango vacío
  \item Rango puntual
  \item Rango de pertenencia
  \item Rango numérico
\end{enumerate}

Es decir, si a partir de una expresión de rango pueden formarse,
simultáneamente, un rango vacío y uno puntual, se tomará el vacío. En caso de
poder formarse dos rangos de pertenencia, se tomará la primera colección para
ejecutar la iteración, y la segunda relación de pertenencia será evaluada para
cada elemento. En caso de poderse formar más de un rango numérico, se tomará su
intersección.

Por último, se agregó el operador de cuantifiación \ingra{count}, o
equivalentemente \ingra{#}. Este operador exige un cuerpo de tipo booleano, y el
resultado de la cuantificación será el número de veces que sea cierta la
expresión del cuerpo para los valores en el rango dado. De esta forma, se
cumplen las equivalencias de los fragmentos de código \ref{lst:forallcount} y
\ref{lst:existcount}.

\begin{gracielacode}[caption=Equivalencia entre cuantificador universal y count, label=lst:forallcount]
(% forall i : T | ~\textit{Rango}(~i) | ~\textit{Cuerpo}~(i) %) ===
    (% count i : T | ~\textit{Rango}(~i) | ! ~\textit{Cuerpo}~(i) %) == 0
\end{gracielacode}

\begin{gracielacode}[caption=Equivalencia entre cuantificador existencial y count, label=lst:existcount]
(% exist i : T | ~\textit{Rango}(~i) | ~\textit{Cuerpo}~(i) %) ===
    (% count i : T | ~\textit{Rango}(~i) | ~\textit{Cuerpo}~(i) %) != 0
\end{gracielacode}

Este operador de cuantificación fue sugerido por Dijkstra~\cite{ewd737}, por lo
cual se consideró apropiado agregarlo al lenguaje inspirado por su GCL.

%%%%%%%%%%%%%%%%%%%%%%%%%%%%%%%%%%%%%%%%%%%%%%%%%%%%%%%%%%%%%%%%%%%%%%%%%%%%%%%%
%% APUNTADORES %%%%%%%%%%%%%%%%%%%%%%%%%%%%%%%%%%%%%%%%%%%%%%%%%%%%%%%%%%%%%%%%%
%%%%%%%%%%%%%%%%%%%%%%%%%%%%%%%%%%%%%%%%%%%%%%%%%%%%%%%%%%%%%%%%%%%%%%%%%%%%%%%%
\section{Apuntadores}

Los apuntadores se declaran como cualquier otra variable dentro del lenguaje
Graciela. El tipo de un apuntador se especifica colocando el tipo que se desea
apuntar, seguido de un asterisco (\ingra{*}). Se pueden declarar apuntadores a
tipos básicos, arreglos, tipos definidos por el programador y a otros
apuntadores. De igual forma se pueden especificar parámetros para funciones y
procedimientos con apuntadores. Cabe acotar que para declarar apuntadores a
tipos cuya sintaxis pueda generar ambigüedad, se permite colocar entre
paréntesis el tipo apuntado, seguido por el asterisco. Un ejemplo de este caso
sería poder declarar un apuntador a un arreglo de enteros, en contraposición a
un arreglo de apuntadores a enteros.

\begin{gracielacode}[caption=Arreglo de apuntadores y apuntador a arreglo, label=lst:arrPtr]
var ptrToArray : (array[10] of int)*
var arrayOfPtr :  array[10] of int *
\end{gracielacode}

Una variable de tipo apuntador puede ser comparada con otra, siempre que los
tipos apuntados sean los mismos, haciendo uso de los operadores de igualdad
(\ingra{==}) y desigualdad (\ingra{!=}). Tambien se puede asignar el valor
de un apuntador a otro, siempre y cuando se cumpla la condición mencionada
anteriormente. Se cuenta con el símbolo reservado del lenguaje
\ingra{null}, que representa a la dirección nula y con el que se pueden
realizar comparaciones y asignaciones a cualquier variables de un
tipo apuntador. No está permitida la aritmética de apuntadores ni el uso de
variables de este tipo en las rutinas \ingra{read} y \ingra{write}.

Cuando se declara una variable de tipo apuntador, ésta contendrá la dirección
nula. Para poder reservar memoria dinámica a la cual apuntar, se puede hacer
uso de la rutina propia del lenguaje \ingra{new}, la cual recibe el apuntador
en cuestión y le asigna una sección de memoria inicializada en cero, del
tamaño del tipo interno. Así mismo, debe utilizarse la rutina propia del
lenguaje \ingra{free}, la cual libera la memoria que apunta la variable que
se le es dada como argumento.

Para poder acceder al contenido de un apuntador, se debe usar el operador
prefijo de desreferencia (\ingra{*}). No se permite desreferenciar un
apuntador cuyo valor sea la dirección nula, y caso de realizarlo, se emite un
error durante la ejecución. De este modo, el fragmento código
\ref{lst:pointer} debe ejecutarse cumpliendo todas las aserciones.

\begin{gracielacode}[caption=Uso de apuntadores, label=lst:pointer]
|[ var p1, p2 : int*; { p1 == null ~\Land~ p2 == null }
 ; new(p1)            { p1 != null ~\Land~ *p1 == 0 }
 ; new(p2)            { p2 != null ~\Land~ *p2 == 0 }
 ; *p1 := 1           { p1 != p2 ~\Land~ *p1 == 1 }
 ; *p2 := *p1 + 1     { p1 != p2 ~\Land~ *p2 == 2 }
 ; free(p1)
 ; free(p2)
 ; p1 := null         { p1 == null }
 ; p2 := p1           { p2 == null }
]|
\end{gracielacode}

%%%%%%%%%%%%%%%%%%%%%%%%%%%%%%%%%%%%%%%%%%%%%%%%%%%%%%%%%%%%%%%%%%%%%%%%%%%%%%%%
%% TEORÍA DE CONJUNTOS %%%%%%%%%%%%%%%%%%%%%%%%%%%%%%%%%%%%%%%%%%%%%%%%%%%%%%%%%
%%%%%%%%%%%%%%%%%%%%%%%%%%%%%%%%%%%%%%%%%%%%%%%%%%%%%%%%%%%%%%%%%%%%%%%%%%%%%%%%
\section{Teoría de Conjuntos}

El lenguaje permite, en las aserciones, el uso de expresiones de la teoría de
conjuntos. Específicamente, se permiten expresiones de tipo Conjunto,
Multiconjunto y Secuencia. En adelante, usaremos el término \textit{colección}
para referirnos a los tres tipos indistintamente. Sólo se permiten colecciones
de un nivel, es decir, no se permiten colecciones de colecciones sino únicamente
de elementos de los tipos básicos (\ingra{int}, \ingra{char}, \ingra{float},
\ingra{boolean}) o de pares ordenados de tipos básicos. Los conjuntos son
representados con los símbolos \ingra|{| y \ingra|}|, los multiconjuntos con
\Lbag{} y \Rbag{} si se usan caracteres UTF-8, o \ingra|{:| y \ingra|:}| si no, y
las colecciones con \Lseq{} y \Rseq{} si se usan caracteres UTF-8, o \ingra|<<| y
\ingra|>>| si no.

Se permite escribir colecciones de dos maneras, por extensión y por comprensión.
En el primer caso, simplemente se nombran los elementos de la colección
separados por comas, existiendo la posibilidad de no especificar elementos para
la colección vacía. Así, el conjunto, multiconjunto y secuencia con los números
1, 2 y 3 pueden escribirse en Graciela, respectivamente, como se muestra en el
fragmento de código \ref{lst:extcollections}.

\begin{gracielacode}[caption=Expresiones de tipos \textit{colección} por extensión, label=lst:extcollections]
{1, 2, 3}
~\Lbag~1, 2, 3~\Rbag~ // ó {:1, 2, 3:} sin UTF-8
~\Lseq~1, 2, 3~\Rseq~ // ó <<1, 2, 3>> sin UTF-8
\end{gracielacode}

En el caso de las colecciones definidas por comprensión, se usa una notación
bastante similar a la de las cuantificaciones. Esta notación consiste en tres
partes escritas entre los símbolos correspondientes a la colección deseada y
separadas por caracteres \ingra{|}. En la primera parte se define una variable
generadora con uno de los tipos permitidos para colecciones. En la segunda, el
rango, se delimita el rango de la variable auxiliar, con las mismas
consideraciones que para las cuantificaciones. La tercera parte, el cuerpo, es
la expresión que será agregada a la colección para cada valor posible de la
variable auxiliar. Así, las mismas colecciones del ejemplo anterior podrían
escribirse, respectivamente, como se muestra en el
fragmento de código \ref{lst:compcollections}.

\begin{gracielacode}[caption=Expresiones de tipos \textit{colección} por comprensión, label=lst:compcollections]
{i : int | 0 < i /\ i < 4 | i}
~\Lbag~i : int | 0 < i /\ i < 4 | i~\Rbag~
  // ó {:i : int | 0 < i /\ i < 4 | i:} sin UTF-8
~\Lseq~i : int | 0 < i /\ i < 4 | i~\Rseq~
  // ó <<i : int | 0 < i /\ i < 4 | i>> sin UTF-8
\end{gracielacode}

También se permite la declaración de variables de tipos colección dentro de los
Tipos de Dato Abstractos. La notación para esto es la misma que para las
declaraciones de tipos básicos, cambiando el tipo usado. Así, para definir,
respectivamente, un conjunto, un multiconjunto, y una secuencia de enteros, se
usa la notación del fragmento de código \ref{lst:declcollections}.

\begin{gracielacode}[caption=Declaración de variables de tipos \textit{colección}, label=lst:declcollections]
var x : set of int;
var y : multiset of int;
var z : sequence of int;
\end{gracielacode}

Además de las colecciones, se agregaron al lenguaje los tipos abstractos de la
teoría de conjuntos \textit{función} y \textit{relación}. En la mayoría de la
literatura que hace uso de la teoría de conjuntos, estos tipos no son más que
conjuntos de pares ordenados, con la restricción adicional de que no pueden
pertenecer a la misma función dos pares con el mismo primer elemento. En el
lenguaje Graciela, para distinguir visualmente estos tipos de los conjuntos
sencillos, y entre ellos, se optó por definir las funciones \ingra{func} y
\ingra{rel}, que toman un conjunto de pares ordenados (\ingra{A}, \ingra{B}),
donde \ingra{A} y \ingra{B} son tipos básicos posiblemente distintos, y
devuelven, respectivamente, la función de \ingra{A} en \ingra{B} y la relación
entre \ingra{A} y \ingra{B} correspondientes al conjunto de pares dado.
Adicionalmente, en el caso en que el conjunto pasado a \ingra{func} contuviese
dos pares ordenados con el mismo primer elemento, se daría un error a tiempo de
ejecución, por incumplirse la definición matemática de \textit{función}. Así, la
función que duplica los números entre el 1 y el 10 puede escribirse como se
muestra en el fragmento de código \ref{lst:funcfunc}, mientras que la relación
entre los números entre el 1.0 y el 10.0 con sus raíces cuadradas positiva y
negativa puede escribirse como se muestra en el fragmento de código
\ref{lst:relfunc}

\begin{gracielacode}[caption=Expresión de tipo \textit{función}, label=lst:funcfunc]
func({ int i | 0 < i /\ i <= 10 | (i, 2*i) })
\end{gracielacode}

\begin{gracielacode}[caption=Expresión de tipo \textit{relación}, label=lst:relfunc]
rel(
  { int i | 0 < i /\ i <= 10 |
    (toFloat(i),  (sqrt(toFloat(i))))
  } union
  { int i | 0 < i /\ i <= 10 |
    (toFloat(i), -(sqrt(toFloat(i))))
  }
)
\end{gracielacode}

Para evaluar una función o relación en un punto dado, es decir, para conocer el
(o los) valor(es) asociado(s) del codominio a un valor del dominio, se usa la
misma notación que para la evaluación de las funciones del lenguaje, es decir,
se escribe la función seguida del valor del dominio en cuestión, entre
paréntesis. En el caso de las funciones, el resultado es un valor del tipo del
codominio, o un error a tiempo de ejecución si el valor usado como argumento no
pertenece al dominio, mientras que para las relaciones, el resultado es siempre
un conjunto, potencialmente vacío, de elementos del tipo del codominio. Puede
verse un ejemplo de ambos casos en el fragmento de código \ref{lst:funcreleval}.

\begin{gracielacode}[caption=Evaluación de \textit{función} y \textit{relación}, label=lst:funcreleval]
{ func ({ (2, 4), (3, 6), (5, 10) }) (3) == 6 };
{ rel  ({ (2, 3), (1, 1), (1, -1) }) (1) == {1, -1} };
\end{gracielacode}

De manera similar a las colecciones, se permite la declaración de funciones y
relaciones entre dos tipos, con la notación mostrada en el fragmento de código
\ref{funcreldecl}.

\begin{gracielacode}[caption=Declaración de variables de tipos \textit{función} y \textit{relación}, label=lst:funcreldecl]
var x : function int -> char;
var y : relation int <-> char;
\end{gracielacode}

%%%%%%%%%%%%%%%%%%%%%%%%%%%%%%%%%%%%%%%%%%%%%%%%%%%%%%%%%%%%%%%%%%%%%%%%%%%%%%%%
%% TIPOS DE DATO %%%%%%%%%%%%%%%%%%%%%%%%%%%%%%%%%%%%%%%%%%%%%%%%%%%%%%%%%%%%%%
%%%%%%%%%%%%%%%%%%%%%%%%%%%%%%%%%%%%%%%%%%%%%%%%%%%%%%%%%%%%%%%%%%%%%%%%%%%%%%%%
\section{Tipos de Dato}

Existen dos clasificaciones de los tipos de dato que pueden definirse en el
lenguaje Graciela, los tipos de dato abstractos o TDA y los tipos de dato que
implementan a un TDA. Estos deben ser definidos dentro del espacio global de
definiciones.

\subsection{Tipos de Dato Abstractos}

Los tipos de dato abstractos pueden declararse con la palabra reservada
\ingra{abstract} seguida de un identificador y, de forma opcional, una secuencia
de variables de tipo dentro de paréntesis y separadas por coma. Podemos separar
la estructura interna de un TDA en tres secciones en las cuales se espeicifcan
las caracteristicas y restricciones del mismo. Primero se debe especificar el
modelo de representación, en el cual se permite declarar contantes y variables
de cualquier tipo, incluyendo tipos de la Teoría de Conjuntos. A continuación se
debe especificar el invariante de representación dentro de un bloque contenido
por los símbolos \ingra|{repinv| y \ingra|repinv}|. Por último, se declaran
todos los procedimientos y funciones abstractas con sus respectivas
precondiciones y poscondiciones.

Dentro de los procedimientos abstractos de un TDA pueden ser declaradas
variables, especialmente variables de tipo de la Teoría de Conjuntos, con el
fin de poder usarlas dentro de la precondición y la poscondición. Así, un tipo
de dato abstracto \ingra{Diccionario} con un procedimiento para agregar un
elemento, puede declararse forma en la que se muestra en el fragmento de código
\ref{lst:tadDicc}.

\begin{widegracielacode}[caption=TAD Diccionario, label=lst:tadDicc]
abstract Diccionario (T0, T1) begin
  const MAX := 100 : int;
  var   conoc      : set of T0;
  var   tabla      : function T0 -> T1;

  {repinv MAX > 0 ~\Land~ #conoc ≤ MAX ~\Land~ conoc == domain(tabla) repinv}

  proc agregar (inout d : Diccionario(T0,T1), in c : T0, in v : T1)
    var conoc~'~ := d.conoc : set of T0;
    var tabla~'~ := d.tabla : function T0 -> T1;

    {pre c ~\Notelem~ d.conoc ~\Land~ #d.conoc < d.MAX pre}

    {post  d.conoc == conoc~'~ union {c}
         ~\Land~ d.tabla == tabla~'~ union func({(c, v)}) post}
end
\end{widegracielacode}

\subsection{Implementación de Tipos de Dato Abstractos}

Aquellos tipos de dato que pretendan implementar un TDA, deben ser declarados
con la palabra reservada \ingra{type}, un identificador y sus variables de
tipo, seguido de la palabra reservada \ingra{implements} y el TDA que se
desea implementar. La especificación interna, es similar a la de un TDA, con la
excepción del invariante y relación de acoplamiento.

En primer lugar, el modelo de representación solo puede albergar constantes y
variables de tipos básicos, arreglos, apuntadores y otros tipos de dato
definidos por el programador. A continuación, deben ser especificados el
invariante de representación y el invariante y relación de acoplamiento. El
invariante de acoplamiento se escribe en un bloque contenido por los símbolos
\ingra|{coupinv| y \ingra|coupinv}|, mientras que la relación de
acoplamiento se escribe dentro de un bloque entre llaves y precedido por la
palabra reservada \ingra{where}. La relación de acoplamiento debe especificar
un mecanismo, a través de asignaciones, con el cual poder materializar todas
las variables de alto nivel, encontradas en el TDA sin excepción. El bloque
\ingra{where} es opcional, siempre y cuando el TDA no tenga variables de alto
nivel, mas no se permite que quede vacío.

Los procedimientos y funciones dentro de un tipo de dato se definen como
cualquier otro fuera de éste. Sin embargo, ciertas condiciones aplican al
momento de especificar el tipo de dato. La primera condición, es que todos
procedimientos y funciones halladas en la especificación del TDA deben
implementarse, aunque esto no evita poder definir otros métodos. La segunda
condición es que, cualquier uso de variables de tipo dentro del cuerpo o
cabezera de un método, está restringido al caso en el que uno de los
parámetros del mismo, sea del tipo de dato dentro del cual se está definiendo
y contenga a su vez las variables de tipo en cuestión.

Todas las variables y constantes presentes en el TDA, estarán disponibles en
cualquier lugar del código como uno de los campos de la
\textit{implementación}, si y solo si, no sea de un tipo de la Teoría de
Conjuntos. El único lugar donde se está permitido hacer uso de las variables
de alto nivel del TDA como campos de la \textit{implementación}, es dentro del
invariante y la relación de acoplamiento. De esta forma, una posible
implementación del TDA \ingra{Diccionario} debería verse de la como se muestra
en el fragmento de código \ref{lst:dicc}

\begin{widegracielacode}[caption=Una posible implementación del TAD Diccionario, label=lst:dicc]
type Dicc (T0, T1) implements Diccionario (T0, T1) begin
  var clave : array [MAX] of T0;
  var valor : array [MAX] of T1;
  var tam   : int;

  {repinv MAX > 0 ~\Land~ 0 ≤ tam ~\Land~ tam ≤ MAX
        ~\Land~ (% ~\Forall~ i : int | 0 ≤ i ~\Land~ i < tam |
              (% ~\Forall~ j : int | 0 ≤ j ~\Land~ j < tam ~\Land~ i != j |
                  clave[i] != clave[j] %) %) repinv}

  {coupinv true coupinv}

  where {
  tabla := func({i : int | 0 ≤ i ~\Land~ i < tam | (clave[i], valor[i])});
  conoc := domain(tabla)
  }

  proc agregar (inout d : Dicc (T0,T1), in c : T0, in v : T1)
    var tam~'~ := d.tam : int;
    var clave~'~        : array[d.MAX] of T0;
    var valor~'~        : array[d.MAX] of T1;

    {pre ¬(% ~\Exist~  i : int | 0 ≤ i ~\Land~ i < d.tam | d.clave[i] == c %)
          ~\Land~ d.tam < d.MAX pre}

    {post d.tam == 1 + tam~'~ ~\Land~ d.clave[tam~'~] == c ~\Land~ d.valor[tam~'~] == v
        ~\Land~ (% ~\Forall~ i : int | 0 ≤ i ~\Land~ i < tam~'~ | d.clave[i] == clave~'~[i]
                                          ~\Land~ d.valor[i] == valor~'~[i] %)
    post}

    |[
     ...
    ]|
end
\end{widegracielacode}

Cabe mencionar que no se pueden declarar variables de un TDA fuera del mismo y
pueden existir multiples implementaciones para un TDA.

\section{Herramientas para el programador}
\subsection{Facilidad de instalación y uso para programadores noveles}
\todo{APT + Homebrew}
\blindtext[1]

\subsection{Mejoras al ambiente de programación}
\todo{Editores}
\blindtext[1]

\subsection{Análisis de rendimiento}
\blindtext[1]
