\chapter{Marco Teórico}
\label{capitulo1}
\lhead{Capítulo 1. \emph{Marco Teórico}}

\section{Programación formal y tripletas de Hoare}

\section{Teoría de conjuntos}

La Teoría de Conjuntos es la rama de la matemática que se encarga del estudio
de colecciones bien formados de objetos, que pueden o no ser de naturaleza
matemática. Esta Teoría surge de la idea original de conjuntos de Georg
Cantor, quien los definió como una colección, finita o infinita, de objetos
definidos y distinguibles, y a su vez los conjuntos pueden ser
considerados en si mismo como un objeto.

A partir de las ideas de Cantor, se comenzó un proceso de axiomatización de la
matemática, con el cual se construyeron en base a los conjuntos, otros objetos
matemáticos como los números, las funciones y otras estructuras. De
de la Teoría de Conjuntos podemos destacar las siguientes estructuras:

\begin{enumerate}[label=\emph{}]
 \litem{Conjunto} Los conjuntos son colecciones, en la cual todos sus
 elementos debes ser distintos entre sí y cuenta con operadores binarios como
 la unión, la intersección y la resta entre conjuntos, así como el operador
 complemento, con el que se obtiene un nuevo conjunto que contiene todos los
 elementos, con respecto a un universo, que el conjunto original no poseía.

 \litem{Multiconjunto} Los multiconjuntos son colecciones, donde cada elemento
 tiene una multiplicidad asociada, es decir, puede contener dos o mas elementos
 que sean idénticos entre sí. También cuenta con operadores las binarias de unión,
 intersección y resta de multicunjuntos, así como con la suma de
 multiconjuntos.

 \litem{Secuencia} Las secuencias son colecciones de objetos enumerados, en las cuales
 se permite la repetición de los mismos. A diferencia del conjunto y el
 multiconjunto, el orden de los elementos si tiene importancia. Formalmente,
 una secuencia puede ser definida como una función entre los números naturales
 y los elementos en sí de la colección.

 \litem{Función} Las funciones son una relación entre un conjunto de entrada (dominio) y
 un conjunto de salida (codominio), con la restricción de que cada elemento
 del dominio se corresponde con un único elemento del codominio, construyendo
 así un mecanismo que permite transformar cualquier elemento del dominio a uno
 del codominio. De igual forma, las funciones pueden modelarse formalmente
 como un conjunto de pares, donde el primer elemento del par pertenece al
 dominio y el segundo al codominio, siguiendo con la restricción de que no
 hayan dos pares cuyo primer elemento sea igual.

 \litem{Relación} Una relación es similar a una función, pero sin la
 restricción de que cada elemento del dominio se corresponda con un solo
 elemento del codominio, si no mas bien, cada elemento se corresponde a un
 conjunto de elementos del codominio.

\end{enumerate}

\section{Tipos de datos abstractos}

\section{Tipos Algebraicos Libres}

En la teoría de tipos, un tipo Algebraico libre es una forma particular de tipo
compuesto que permite definir tipos producto y tipos suma, además de tipos que
son una combinación de ambos. Los valores de un tipo producto suelen contener
varios sub-valores de distintos tipos, y el conjunto de valores posibles es el
producto cartesiano de los conjuntos de cada sub-tipo, de ahí la denominación de
tipo <<producto>>. Por otro lado, los tipos suma definen varias \textit{clases}
tales que  los valores de un tipo suma sólo pueden pertenecer a una clase a la
vez, de modo que el conjunto de valores posibles para un tipo suma es la unión
de los valores posibles para cada clase, notando que la cardinalidad de este
conjunto será la suma de los conjuntos de las clases, debido a que son
disjuntos.

El ejemplo motivador en la mayoría de los cursos que hablan sobre Tipos
Algebraicos Libres (TALs) es el del árbol binario. Un lenguaje que soporte TALs
aceptaría una definición como la siguiente.

$$ \textbf{freetype}\ Tree(e)\ = Leaf(e)\ |\ Node(Tree\ e,\ Tree\ e) $$

Este TAL es el tipo suma suma de las clases
$Leaf(e)$ y $Node(Tree\ (e),\ Tree\ (e))$,
y esta última es, a su vez, un tipo producto de $(Tree(e))$ y $(Tree(e))$.

Otra característica importante de los TALs, que los hace muy útiles, es la
posibilidad de verificar a cuál clase pertenece un valor de un tipo producto,
con el predicado \textbf{is}, y la de extraer los sub-valores de un tipo suma
con la instrucción \textbf{match}. El primero se puede usar, siguiendo con el
ejemplo motivador, dentro de un condicional como en el ejemplo \ref{ifis},
mientras que el segundo, como se muestra en el ejemplo \ref{matching}, se puede
usar una vez se tiene certeza de que el valor es de una clase en particular.

\begin{alignat}{3}
&\boldsymbol{if}\ && v\ \boldsymbol{is}\ Leaf && \rightarrow write("leaf") \nonumber \\
&\boldsymbol{[]}\ && v\ \boldsymbol{is}\ Node && \rightarrow write("node") \label{ifis} \\
&\boldsymbol{fi} \nonumber
\end{alignat}

\begin{alignat}{4}
&\boldsymbol{if}\ && v\ \boldsymbol{is}\ Leaf && \rightarrow &&\ var\ elem : int                   \nonumber \\
&                 &&                          &&           ; &&\ v\ \boldsymbol{match}\ Leaf(elem) \nonumber \\
&                 &&                          &&           ; &&\ write (elem)                      \label{matching} \\
&\boldsymbol{[]}\ && \ldots                                                                        \nonumber \\
&\boldsymbol{fi}                                                                                   \nonumber
\end{alignat}

Adicionalmente, es común definir la instrucción \textbf{matches} \todo{ravelo}
como la aplicación secuenciada del predicado \textbf{is} y la instrucción
\textbf{match} de la clase apropiada. El ejemplo \ref{matches} muestra esto,
como una combinación de los ejemplos anteriores, pero sin la necesidad de
declarar las variables para los sub-valores.

\begin{alignat}{3}
&\boldsymbol{if}\ && v\ \boldsymbol{matches}\ Leaf(elem)         && \rightarrow write(elem) \nonumber \\
&\boldsymbol{[]}\ && \ldots                                                        \label{matches} \\
&\boldsymbol{fi} \nonumber
\end{alignat}

% \todo{http://foldoc.org/algebraic%20data%20type}
% \todo{\verb{http://haskell.cs.yale.edu/wp-content/uploads/2011/02/history.pdf}}

\section{Lógica de Separación}

La Lógica de Separación fue desarrollada por John Raynolds, Hongseok Yang y
Samin Ishtiaq \todo{referencia a sus papers} como una extensión de la lógica de
Hoare que permite razonar sobre el acceso a datos en la memoria de una
computadora y como estos pueden mutar. Tal como lo explica Peter O'Hearn en
\todo{ref al paper}, la Lógica de Separación se basa en la \textit{conjunción
separadora} P * Q, que significa que P y Q apuntan a porciones de memorias
disjuntas. De esta forma se pueden escribir predicados que permitan realizar
verificaciones sobre estructuras que hacen uso de apuntadores, como por
ejemplo verificar que una estructura es un árbol binario.


\begin{align*}%\label{eq:treesl}
  tree(E) \Longleftrightarrow\ &\boldsymbol{if}\ isatom?(E)\ \boldsymbol{then}\ emp\\
             &\boldsymbol{else}\ \exists xy.\ E\mapsto[l:\ x,\ r:\ y]\ ∗\ tree(x)\ ∗\ tree(y)
\end{align*}


Un árbol binario se caracteriza por el hecho de que cada nodo debe estar en un
espacio de memoria distinto a los demás, lo que asegura que no existan ciclos.
En el ejemplo anterior \todo{ref?}, asumiendo que el predicado \textbf{isatom?} puede realizar
la distinción entre si es un valor atómico (entero, caracter, ...) o una
referencia a memoria, se procede a verificar si la estructura E es un átomo.
En caso positivo se devuelve el valor \textbf{emp}, el cual representa al heap
vacío, es decir, el heap en donde no existen celdas reservadas. Si por el
contrario E no es un átomo, entonces se verifica para cada hijo que sea un
árbol binario, cubriendo así el caso recursivo. El operador \todo{nombre?}
$E\mapsto x$ significa que E apunta al fragmento de memoria x. En este caso se
hace uso de este operador de la forma $E\mapsto [l: x, r: y]$ que significa que
los hijos izquierdo y derecho se ubican en los fragmentos de memorias $x$ y $y$
respectivamente, y no es mas que una abreviación de $(E\mapsto x) * (E+1\mapsto x)$.

Aunado a esto, la lógica de separación permite hacer pruebas de la forma $s,h
\vDash P$ donde s es un valor almacenado, h es un fragmento de memoria o heap
y P es una aserción que se realizar sobre s y h. También se cuenta con el operador
\textit{implicación separadora} $P-*\ Q$ que se entiende como, si para un heap dado
se cumple P, entonces al extender este primer heap con otro heap disjunto se
debe cumplir Q, y de manera formal, $s,h \vDash P-*\ Q$ se cumple si para todo h'
disjunto de h $s,h' \vDash P$ y $s,h\Union h' \vDash Q$.
