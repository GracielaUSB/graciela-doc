\chapter*{Introducción}
\label{intro}
\lhead{\emph{Introducción}}

\section*{Planteamiento del problema}


\section*{Justificación}


\section*{Antecedentes}


\section*{Objetivos}


\subsection*{Objetivo General}

Extender el compilador del lenguaje Graciela, una variante del GCL (Guarded
Command Language) de Dijkstra, implantado por Araujo y Jiménez. Esta extensión
deberá soportar tipos de datos definidos por el programador, apuntadores
explícitos, manejo de memoria automático, y un mecanismo de aserciones
verificables que soporte estas nuevas capacidades.

\subsection*{Objetivos Específicos}
\begin{itemize}
  \item Revisión bibliográfica relacionada con implantación de compiladores,
  sistemas de tipos y semántica axiomática.

  \item Especificación formal de la extensión a Graciela a implantar.

  \item Evaluación de herramientas que faciliten la construcción del ambiente de
  ejecución final para manejo automático de memoria.

  \item Implantación del compilador de la extensión de Graciela a código de
  máquina nativo.

  \item Extensión del manual de usuario con las nuevas funcionalidades.
\end{itemize}

\section*{Organización del Trabajo}


