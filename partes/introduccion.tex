\chapter*{Introducción}
\label{intro}
\lhead{\emph{Introducción}}

\section*{Planteamiento del problema}

\section*{Justificación}

\section*{Antecedentes}

El lenguaje  GCL (Guarded Command Language), diseñado por Edsger Dijkstra,
cuenta con otras implementaciones además del compilador para el lenguaje
Graciela. Entre estos implementaciones, se encuentra el lenguaje GaCeLa
\todo{referencia}, el cual tiene especial importancia para el presente
proyecto, ya que es un lenguaje que fue desarrollado previamente por
estudiantes y profesores de la Universidad Simón Bolívar, cuya intención era
introducir a los estudiantes de Algoritmos y Estructuras I a la programación
formal, de manera que pudieran desarrollar programas a partir de su
especificación formal. El lenguaje GaCeLa permite declarar procedimientos,
funciones y tipos algebraicos. Sin embargo, este lenguaje cayo en desuso a
partir del año 2010 y la documentación disponible es limitada.

Además de GaCeLa existen otras implementaciones ajenas a la Universidad Simón
Bolívar, entre los cuales se encuentran el modulo para el lenguaje Perl
llamado Commands::Guarded y GCL 1.2. El segundo un interpretador escrito en
Prolog y C que posee muchas restricciones en cuanto a los arreglos y
aserciones, y no soporta la declaración de procedimientos y funciones.


\section*{Objetivo General}

Extender el compilador del lenguaje Graciela, una variante del GCL (Guarded
Command Language) de Dijkstra, implantado por Araujo y Jiménez. Esta extensión
deberá soportar tipos de datos definidos por el programador, apuntadores
explícitos, manejo de memoria automático, y un mecanismo de aserciones
verificables que soporte estas nuevas capacidades.

\section*{Objetivos Específicos}
\begin{itemize}
  \item Revisión bibliográfica relacionada con implantación de compiladores,
  sistemas de tipos y semántica axiomática.

  \item Especificación formal de la extensión a Graciela a implantar.

  \item Evaluación de herramientas que faciliten la construcción del ambiente de
  ejecución final para manejo automático de memoria.

  \item Implantación del compilador de la extensión de Graciela a código de
  máquina nativo.

  \item Extensión del manual de usuario con las nuevas funcionalidades.
\end{itemize}

\section*{Organización del Trabajo}

El presente trabajo se divide principalmente en dos capítulos, uno para el
desarrollo del proyecto y un otro para los resultados obtenidos. En el
desarrollo del proyecto, se detalla el estudio que se realizó de las
recomendaciones del proyecto anterior y las consideraciones que se tomaron en
cuenta para lograr estas recomendaciones y los objetivos especificados para
este proyecto, incluyendo también los objetivos adicionales que surgieron
durante el desarrollo del mismo. En segundo lugar, se encuentra un capítulo
dedicado a mostrar los resultados del proyecto, donde se describe a detalle la
sintaxis final incorporada al lenguaje Graciela, así como las herramientas
y utilidades que se desarrollaron para el uso de dicho lenguaje.

