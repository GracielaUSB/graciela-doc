\chapter*{Introducción}
\label{intro}
\setchapter{\emph{Introducción}}

Históricamente, y por varios años, en los cursos de Laboratorio de Algoritmos I
y II dictados en la Universidad Simón Bolívar, el lenguaje de programación
utilizado para la instrucción fue GaCeLa, basado en el lenguaje GCL de Edsger
Dijkstra y desarrollado, junto con su compilador, dentro de la Universidad. Este
compilador no generaba código nativo para la plataforma destino, sino que
transformaba el programa a compilar en otro programa escrito en el lenguaje
Java, idealmente con la misma semántica del original. Este programa generado
debía ser ejecutado usando la máquina virtual de Java (JVM o \textit{Java
Virtual Machine} en inglés).

Desafortunadamente, dependiendo del programa en GaCeLa compilado, el programa
generado en Java presentaba errores que no eran reportados sino hasta que se
intentaba ejecutarlo dentro de la JVM, forzando a los estudiantes a modificar
manualmente este archivo generado mecánicamente a fin de poder ejecutarlo,
actividad para la cual claramente no estaban preparados, ni era objeto de
evaluación de los cursos en cuestión. Adicionalmente, no se contaba con el
personal suficiente para mantener el compilador y reparar este tipo de errores a
nivel del compilador en lugar de en el código generado, por lo cual los
profesores que dictaban estas materias cambiaron el lenguaje de instrucción por
otros lenguajes, entre ellos Pascal, Modula 2, JML y Python, cayendo en desuso
al rededor del año 2010. Actualmente, la documentación disponible para GaCeLa es
muy limitada.

En los cursos teóricos de Algoritmos I y II de la Universidad Simón Bolívar, por
otro lado, el lenguaje de instrucción sigue siendo GCL. En consecuencia, los
estudiantes de estos cursos teóricos, que deben ser inscritos simultáneamente
con el laboratorio correspondiente, se ven forzados a hacer una traducción
mental entre los conceptos estudiados en la teoría y los ejercicios elaborados
en el laboratorio.

Durante el año 2015, Joel Araujo y José Luis Jiménez desarrollaron, como
Proyecto de Grado, un nuevo lenguaje, Graciela, y su respectivo compilador. Este
lenguaje cuenta con los tipos básicos \ingra{int}, \ingra{char}, \ingra{float} y
\ingra{boolean}, y arreglos de estos, ofrece instrucciones de lectura,
escritura, asignación, selección y repetición, permite insertar aserciones entre
las instrucciones (posiblemente haciendo uso de cuantificaciones) y permite la
definición de funciones y procedimientos (con los modos \textit{in},
\textit{in-out}, \textit{out} y \textit{ref} para el pase de parámetros). El
compilador desarrollado por Araujo y Jiménez genera código intermedio LLVM que
posteriormente es convertido en código nativo para la plataforma deseada
haciendo uso del \textit{Back-End} de LLVM.

El lenguaje de Araujo y Jiménez tiene todas las características necesarias para
el primer curso práctico de Algoritmos, pero, como lo indican en las
recomendaciones de su Proyecto de Grado, es necesario extender el lenguaje y el
compilador para lograr que estos sean útiles para el segundo curso práctico de
Algoritmos. Específicamente, recomiendan agregar al lenguaje el tipo
\textit{apuntador} y las instrucciones necesarias para manipularlos, y la
posibilidad de crear tipos de dato estructurados con comportamientos definidos
a través de aserciones invariantes.

Así, es evidente la necesidad de agregar estas extensiones al lenguaje Graciela
y a su compilador, dado que son cruciales para el segundo curso práctico de
Algoritmos, concernido precisamente con el manejo de apuntadores y la definición
de tipos estructurados con comportamientos abstractos. Estas extensiones estarán
siempre guiadas por la especificación del lenguaje GCL siempre que esto sea
posible, para continuar la promoción del aprendizaje de la programación usando
métodos formales en apoyo con los cursos teóricos dictados usando el lenguaje
GCL.

\section*{Otros antecedentes}

Además de GaCeLa y Graciela, existen otras implementaciones de las ideas del
lenguaje GCL ajenas a la Universidad Simón Bolívar, entre las cuales se
encuentran  \texttt{Commands::Guarded}~\cite{perlguarded} y GCL
1.2~\cite{gclonetwo}. El primero es un módulo para el lenguaje Perl que explota
la idea de agregarle precondiciones y poscondiciones verificables a cada
instrucción, mientras que el segundo es un interpretador escrito en Prolog y C
que impone muchas restricciones sobre arreglos y aserciones, y no soporta la
declaración de procedimientos y funciones.

\section*{Objetivo General}

Extender el compilador del lenguaje Graciela, una variante del GCL (Guarded
Command Language) de Dijkstra, implantado por Araujo y Jiménez. Esta extensión
deberá soportar tipos de dato definidos por el programador, apuntadores
explícitos, manejo de memoria automático, y un mecanismo de aserciones
verificables que soporte estas nuevas capacidades.

\section*{Objetivos Específicos}
\begin{itemize}
  \item Revisión bibliográfica relacionada con implantación de compiladores,
  sistemas de tipos y semántica axiomática.

  \item Especificación formal de la extensión a Graciela a implantar.

  \item Evaluación de herramientas que faciliten la construcción del ambiente de
  ejecución final para manejo automático de memoria.

  \item Implantación del compilador de la extensión de Graciela a código de
  máquina nativo.

  \item Extensión del manual de usuario con las nuevas funcionalidades.
\end{itemize}

\section*{Organización del Trabajo}

Este Proyecto de Grado se presenta en cuatro capítulos. En el primero, se
establece un marco teórico al cual se hace referencia en los capítulos
subsiguientes, y que permite establecer un nivel de abstracción apropiado para
estos. En el segundo, similarmente, se establece un marco tecnológico, en el
cual se exponen las herramientas que permitieron desarrollar el proyecto en su
forma actual. En el tercer capítulo, el desarrollo del proyecto, se detalla el
estudio que se realizó de las recomendaciones dadas por Araujo y Jiménez y las
consideraciones que se tomaron en cuenta para cumplir estas recomendaciones y
los objetivos especificados para este proyecto, incluyendo también el proceso de
desarrollo asociado a los objetivos adicionales que surgieron durante el
desarrollo del mismo. En el cuarto capítulo se exponen los resultados del
proyecto, describiendo con detalle la sintaxis definitiva incorporada al
lenguaje Graciela y la explicación de la semántica de estas extensiones, así
como las herramientas y utilidades que se desarrollaron para facilitar el uso de
dicho lenguaje.
