\documentclass[letterpaper,11pt]{article}

\usepackage[spanish]{babel}
\usepackage[utf8]{inputenc}
\usepackage{graphicx}
\usepackage[top=2cm, left=2cm, right=2cm, bottom=2cm]{geometry}
\linespread{1.5}

\DeclareGraphicsExtensions{.jpg,.pdf,.mps,.png}

\begin{document}

\title{CI-XXXX \\ Nombre de la Materia\\ Tarea X \\}
\author{Alejandro Machado 07-41138 \\ Universidad Simón Bolívar}
\date{\today}
\maketitle

\thispagestyle{empty}
\pagestyle{empty}

Aquí se escribe la tarea. Uno puede escribir lo que quiera, siempre y cuando
tenga que ver con la materia en cuestión y responda al enunciado de la tarea.
Esto, por ejemplo, responde al enunciado: ``Escriba un formato genérico \LaTeX\
para tareas en la Universidad Simón Bolívar''.

\end{document}
